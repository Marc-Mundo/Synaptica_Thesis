\section*{About Synaptica B.V.}

\subsection*{Company Overview}

Founded by Dr.\ Marijn Martens, Synaptica B.V. is a pioneering medical-scientific research and development company specializing in epilepsy.
Synaptica employs biologically realistic spiking brain circuit models to perform advanced epilepsy calculations.
Dr.\ Martens, a graduate cum laude from Donders Graduate School for Cognitive Neuroscience in 2010, has leveraged his
extensive academic and startup experience to offer innovative diagnostic and therapeutic tools based on digital biomarkers.

\subsection*{Recent Achievements}

\begin{itemize}
    \item \textbf{2024:} Through the EuroCC Open Call Application, Synaptica was awarded access to Snellius,
          the largest supercomputer in the Netherlands. This resource is enabling the company to perform crucial protein
          folding simulations of mutated ion channels, aiding in the prediction of functional consequences associated with channelopathies, such as epilepsy.

    \item \textbf{2023:} Synaptica secured a two-year R\&D MIT AI research grant in collaboration with SME Artinis Medical Systems.
          This partnership focuses on developing a computer model that simulates neuron activity in humans and an fNIRS-based epilepsy headset
          for real-time brain activity monitoring. These tools aim to enhance the accuracy of epilepsy diagnosis and treatment.
\end{itemize}

\subsection*{List of Publications}

\begin{itemize}
    \item \textbf{2023} Human cortical spheroids with a high diversity of innately developing brain cell types, Kim de Klein, (\dots), M.B. Martens et al., \textit{Stem Cell Research \& Therapy}
    \item \textbf{2021} Two-Minute Walking Test With a Smartphone App for Persons With Multiple Sclerosis: Validation Study, Pim van Oirschot, (\dots), M.B. Martens et al., \textit{JMIR Formative Research}
    \item \textbf{2020} Key role for lipids in cognitive symptoms of Schizophrenia, Dorien Maas, (\dots), M.B. Martens et al, \textit{Translational Psychiatry}
    \item Numerous additional publications from 2020 to 2010, including significant contributions to the fields of cognitive neuroscience and digital health technology, emphasizing the genetic and molecular landscapes of various neurological conditions.
\end{itemize}

\noindent Synaptica continues to be at the forefront of neuroscientific research, pushing the boundaries of technology and science to better understand and treat neurological disorders.
