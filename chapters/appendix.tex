\chapter{Appendix}

This chapter holds the appendix of the thesis.

\section{Cellular dynamics: equations and parameters}
\textbf{Basket cells:}
The transient sodium current was described by
\[
    I_{Na,I} = g_{Na,I} m_{\infty}^3 h (V_I - E_{Na,I}),
\]
where
\[
    m_{\infty} = \frac{\alpha_m}{\alpha_m + \beta_m},
\]
and
\[
    \alpha_m(V_I) = -0.1 \frac{(V_I + 35)}{e^{-0.1(V_I+35)} - 1}, \quad \beta_m(V_I) = 4e^{-\frac{(V_I+60)}{18}}.
\]

The inactivation variable \( h \) obeyed the first-order kinetics:
\[
    \frac{dh}{dt} = \phi(\alpha_h (1 - h) - \beta_h h),
\]
where
\[
    \alpha_h(V_I) = 0.07 e^{-\frac{(V_I+58)}{20}}, \quad \beta_h(V_I) = \frac{1}{e^{- 0.1(V_I+28)} + 1}.
\]

The delayed rectifier potassium current was described by
\[
    I_{K,I} = g_{K,I} n^4 (V_I - E_{K,I}),
\]
where the activation variable \( n \) obeyed the following equation:
\[
    \frac{dn}{dt} = \phi(\alpha_n (1 - n) - \beta_n n),
\]
with
\[
    \alpha_n(V_I) = -0.01 \frac{(V_I + 34)}{e^{-0.1(V_I+34)} - 1}, \quad \beta_n(V_I) = 0.125 e^{-\frac{(V_I+44)}{80}}.
\]

For the experiments, the following parameters were used: \( g_{Na,I} = 35 \)
mS/cm\(^2\), \( g_{K,I} = 9 \) mS/cm\(^2\), \( E_{Na,I} = 55 \) mV, \( E_{K,I}
= -90 \) mV, \( \phi = 5 \).\pagebreak

\todo[inline]{fix the equations for olm cells, add them for pyr cells, double check GPT output}
\noindent
\textbf{O-LM cells:}
The channel currents were described by
\[
    I_{Na,o} = g_{Na,o}m^3h(V_O - E_{Na,O}),
\]
where \( m \) obeys
\[
    \frac{dm}{dt} = \alpha_m(1 - m) - \beta_m m,
\]
with
\[
    \alpha_m = \frac{-0.1(V_O + 38)}{e^\frac{- (V_O+38)}{10} - 1}, \quad \beta_m = 4e^{-\frac{(V_O+65)}{18}},
\]
and \( h \) obeys
\[
    \frac{dh}{dt} = \alpha_h(1 - h) - \beta_h h,
\]
with
\[
    \alpha_{h}(V_o) = 0.07e^{-\frac{V_o+63}{20}}, \quad \beta_{h}(V_o) = \frac{1}{1 + e^{-0.1(V_o+33)}},
\]
Similarly,
\[
    I_{K,o} = g_{K,o} n^4 (V_o - E_{K,o}),
\]
\[
    \frac{dn}{dt} = \alpha_n(1 - n) - \beta_n n,
\]
\[
    \alpha_n(V_o) = 0.018\left(\frac{V_o - 25}{1 - e^{-\frac{V_o - 25}{25}}}\right), \quad \beta_n(V_o) = 0.0036\left(\frac{V_o - 35}{1 - e^{\frac{V_o - 35}{12}}}\right),
\]
\[
    I_{h,o} = g_{h,o}r(V_o - E_{h,o}),
\]
\[
    \frac{dr}{dt} = \frac{r_{\infty} - r}{\tau_r},
\]
\[
    \tau_{r_{o}}(V) = \frac{1}{1 + e^{\frac{V + 84}{8.5}}} + r_{\infty}(V) = \frac{e^{-14.59(V + 60)} + e^{-1.87\cdot1.87(V+60)}}{1 + e^{\frac{V+84}{8.5}}},
\]
\[
    I_{A,o} = g_{A,o}ab(V_o - E_{A,o}),
\]
\[
    \frac{da}{dt} = \frac{a_{\infty} - a}{\tau_a},
\]
\[
    a_{\infty}(V_o) = \frac{1}{1 + e^{\frac{-V_o + 14}{16.8}}}, \quad \tau_a(V_o) = 5,
\]
\[
    \frac{db}{dt} = \frac{b_{\infty} - b}{\tau_b},
\]
\[
    b_{\infty}(V_o) = \frac{1}{1 + e^{\frac{V_o + 71}{7.3}}}, \quad \tau_b(V) = \left(\frac{0.00002(V_o - 90)}{0.2e^{-8(V_o + 90)} + 0.2e^{-0.1(V_o + 90)}}\right) + 1.
\]
We used \( g_{Na,o} = 30 \) mS/cm\(^2\), \( g_{K,o} = 23 \) mS/cm\(^2\), \(
g_{h,o} = 12 \) mS/cm\(^2\), \( g_{A,o} = 16 \) mS/cm\(^2\), \( E_{Na,o} = 90
\) mV, \( E_{K,o} = -100 \) mV, \( E_{h,o} = -32.9 \) mV, \( E_{A,o} = -90 \)
mV.\pagebreak

\section{Additional data}

\section{Additional figures}

\section{Additional tables}

\todo[inline]{Add additional tables here for all the celltypes and for the noise with all the parameters}

\begin{table}[htbp]
    \centering
    \caption[Synaptic Parameters for the Connectivity Between Neurons in the Model]{Synaptic Parameters for the Connectivity Between Neurons in the Model: Pre- and postsynaptic receptor types are given for each cell type.
        The time constants \(\tau_1\) and \(\tau_2\) are in milliseconds.
        \(\tau_1\) is the rise time constant, the time it takes for synaptic conductance to increase from baseline to peak.
        \(\tau_2\) is the decay time constant, the time it takes for the conductance to decrease from peak to baseline.
        The conductance indicates the strength of the synaptic connection and its ability to conduct ionic current across the postsynaptic membrane.
        This influences the extent to which the synaptic input can depolarize the postsynaptic neuron and is in nanoSiemens (nS).}
    \begin{tabular}{lllccc}
        \hline
        Presynaptic & Postsynaptic & Receptor & \(\tau_1\) (ms) & \(\tau_2\) (ms) & Conductance (nS) \\
        \hline
        Pyramidal   & Pyramidal    & AMPA     & 0.05            & 5.3             & 0.02             \\
        Pyramidal   & Pyramidal    & NMDA     & 15              & 150             & 0.004            \\
        Pyramidal   & Basket       & AMPA     & 0.05            & 5.3             & 0.36             \\
        Pyramidal   & Basket       & NMDA     & 15              & 150             & 1.38             \\
        Pyramidal   & OLM          & AMPA     & 0.05            & 5.3             & 0.36             \\
        Pyramidal   & OLM          & NMDA     & 15              & 150             & 0.72             \\
        Basket      & Pyramidal    & GABA-A   & 0.07            & 9.1             & 0.72             \\
        Basket      & Basket       & GABA-A   & 0.07            & 9.1             & 4.5              \\
        Basket      & OLM          & GABA-A   & 0.07            & 9.1             & 0.0288           \\
        OLM         & Pyramidal    & GABA-A   & 0.2             & 20              & 72               \\
        MS          & Basket       & GABA-A   & 20              & 40              & 1.6              \\
        MS          & OLM          & GABA-A   & 20              & 40              & 1.6              \\
        \hline
    \end{tabular}\label{tab:synaptic_parameters_copy}
\end{table}

\begin{table}[htbp]
    \centering
    \caption[NetStim Parameters Summary]{NetStim Parameters Summary. The ``Number'' of spikes for each stimulus is calculated using the formula \((1e3 / \text{Interval}) \cdot h.tstop\), adjusting based on the specified interval for each stimulus.
        This results in a target number of spikes over the simulation period.}
    \begin{tabular}{lcccl}
        \hline
        \textbf{Stimulus} & \textbf{Interval (ms)} & \textbf{Noise} & \textbf{Weight (e-3)}                   & \textbf{Target} \\
        \hline
        Pyr 1             & 1                      & 1              & 0.05                                    & somaAMPAf       \\
        Pyr 2             & 1                      & 1              & \(0.05 \cdot \text{pyr\_noise\_scale}\) & Adend3AMPAf     \\
        Pyr 3             & 1                      & 1              & 0.012                                   & somaGABAf       \\
        Pyr 4             & 1                      & 1              & 0.012                                   & Adend3GABAf     \\
        Pyr 5             & 100                    & 1              & \(6.5 \cdot \text{pyr\_noise\_scale}\)  & Adend3NMDA      \\ \hline
    \end{tabular}
\end{table}\label{table:netstimparams_reduced}