\chapter{Appendix}\label{ch:appendix}
\section{Cellular dynamics: equations and parameters}
The mathematical formulations of the cell types below are based on the work of
\textcite{sanjayImpairedDendriticInhibition2015} and have been implemented as
published in their model release on GitHub in the relevant NEURON mod files.
\noindent \textbf{Basket cells:} The transient sodium current was described by
\[
    I_{Na,I} = g_{Na,I} m_{\infty}^3 h (V_I - E_{Na,I}),
\]
where
\[
    m_{\infty} = \frac{\alpha_m}{\alpha_m + \beta_m},
\]
and
\[
    \alpha_m(V_I) = -0.1 \frac{(V_I + 35)}{e^{-0.1(V_I+35)} - 1}, \quad \beta_m(V_I) = 4e^{-\frac{(V_I+60)}{18}}.
\]

The inactivation variable \( h \) obeyed the first-order kinetics:
\[
    \frac{dh}{dt} = \phi(\alpha_h (1 - h) - \beta_h h),
\]
where
\[
    \alpha_h(V_I) = 0.07 e^{-\frac{(V_I+58)}{20}}, \quad \beta_h(V_I) = \frac{1}{e^{- 0.1(V_I+28)} + 1}.
\]

The delayed rectifier potassium current was described by
\[
    I_{K,I} = g_{K,I} n^4 (V_I - E_{K,I}),
\]
where the activation variable \( n \) obeyed the following equation:
\[
    \frac{dn}{dt} = \phi(\alpha_n (1 - n) - \beta_n n),
\]
with
\[
    \alpha_n(V_I) = -0.01 \frac{(V_I + 34)}{e^{-0.1(V_I+34)} - 1}, \quad \beta_n(V_I) = 0.125 e^{-\frac{(V_I+44)}{80}}.
\]

For the experiments, the following parameters were used: \( g_{Na,I} = 35 \)
mS/cm\(^2\), \( g_{K,I} = 9 \) mS/cm\(^2\), \( E_{Na,I} = 55 \) mV, \( E_{K,I}
= -90 \) mV, \( \phi = 5 \).\pagebreak

\noindent
\textbf{O-LM cells:}
The channel currents were described by
\[
    I_{Na,o} = g_{Na,o}m^3h(V_O - E_{Na,O}),
\]
where \( m \) obeyed
\[
    \frac{dm}{dt} = \alpha_m(1 - m) - \beta_m m,
\]
with
\[
    \alpha_m = \frac{-0.1(V_O + 38)}{e^\frac{- (V_O+38)}{10} - 1}, \quad \beta_m = 4e^{-\frac{(V_O+65)}{18}},
\]
and \( h \) obeyed
\[
    \frac{dh}{dt} = \alpha_h(1 - h) - \beta_h h,
\]
with
\[
    \alpha_n(V_O) = 0.07e^{\frac{- (V_O + 63)}{20}}, \quad \beta_n(V_O) = \frac{1}{1 + e^\frac{-0.1(V_O+33)}{10}},
\]
Similarly,
\[
    I_{K,O} = g_{K, O}n^4(V_O - E_{K,O}),
\]
\[
    \frac{dn}{dt} = \alpha_n(1 - n) - \beta_n n,
\]
\[
    \alpha_n(V_O) = \frac{0.018(V_O - 25)}{1 - e^{-\frac{V_O - 25}{25}}}, \quad \beta_n(V_O) = \frac{0.0036(V_O - 35)}{e^{\frac{V_O - 35}{12}} - 1}
\]
\[
    I_{h,O} = g_{h,O}r(V_O - E_{h,O}),
\]
\[
    \frac{dr}{dt} = \frac{r_{\infty} - r}{\tau_r},
\]
\[
    r_{\infty}(V_O) = \frac{1}{1 + e^{-\frac{V_O + 84}{10.2}}}, \quad \tau_r(V_O) = \frac{1}{e^{-14.59 - 0.086V_O} + e^{-1.87 + 0.0701V_O}};
\]
\[
    I_{A,O} = g_{A,O}ab(V_O - E_{A,O}),
\]
\[
    \frac{da}{dt} = \frac{\alpha_{\infty} - a}{\tau_a},
\]
\[
    \alpha_{\infty}(V_O) = \frac{1}{1 + e^{\frac{- (V_O+14)}{16.6}}}, \quad \tau_a(V_O) = 5,
\]
\[
    \frac{db}{dt} = \frac{b_{\infty} - b}{\tau_b},
\]
\[
    b_{\infty}(V_O) = \frac{1}{1 + e^{\frac{V_O+71}{7.3}}}, \quad \tau_b(V_O) = \frac{1}{\frac{0.000009}{e^{\frac{V_O-26}{18.5}}} + \frac{0.014}{0.2+e^{\frac{- (V_O+70)}{11}}}}.
\]
For the experiments, the following parameters were used: \(g_{Na,o} = 30\)
mS/cm\(^2\), \(g_{K,O} = 23\) mS/cm\(^2\), \(g_{h,O} = 12\) mS/cm\(^2\),
\(g_{A,O} = 16\) mS/cm\(^2\), \(E_{Na,O} = 90\) mV, \(E_{K,O} = -100\) mV,
\(E_{h,O} = -32.9\) mV, \(E_{A,O} = - 90\) mV.\pagebreak

\noindent
\textbf{Pyramidal cells:}
\(I_{\text{conn}, E_k}\) was the current due to electrical coupling between compartments, which was given by
\[
    Iconn_{E,k} = g_{k,j+1}(V_{E,j+1} - V_{E,j}) + g_{k,j}(V_{E,j-1} - V_{E,j})
\]
with the coupling conductance given by
\[
    g_{k,j} = \frac{r_k r_j^2}{R_a L_k (L_k r_j^2 + L_j r_k^2)}
\]
where \(L_k\) and \(r_k\) was the length and radius of the compartment \(k\)
respectively (note the need of units conversion in order to get \(g_{k,j}\) in
mS/cm\(^2\)). The ionic currents were given by:
\begin{align*}
    I_{Na,E_k} & = g_{Na,E_k} m^3 h i (V_{E_k} - E_{Na,E}) , \\
    I_{K,E_k}  & = g_{K,E_k} n^4 (V_0 - E_{K,E}) ,           \\
    I_{h,E_k}  & = g_{h,E_k} r (V_{E_k} - E_{h,E}) ,         \\
    I_{A,E_k}  & = g_{A,E_k} ab (V_{E_k} - E_{A,E}) ,
\end{align*}
and the gating variables obeyed equations of the form
\[
    \frac{dx}{dt} = \frac{x_{\infty} - x}{\tau_x},
\]
where \(x = m, h, i, n, r, a, b\) and
\[
    m_{\infty} = \frac{\alpha_m}{\alpha_m + \beta_m}, \quad \tau_m = \max\left(0.2, \frac{0.5}{\alpha_m + \beta_m}\right),
\]
\begin{align*}
    \alpha_m(V_{E_k})   & = \frac{0.4(V_{E_k} + 30)}{1 - e^{-\frac{V_{E_k} + 30}{7.2}}},               & \beta_m(V_{E_k}) & = \frac{0.124(V_{E_k} + 30)}{e^{\frac{V_{E_k} + 30}{7.2}} - 1}, \\
    h_{\infty}(V_{E_k}) & = \frac{1}{1 + e^{\frac{V_{E_k} + 50}{4}}},                                  & \tau_h           & = \max\left(0.5, \frac{0.5}{\alpha_h + \beta_h}\right),         \\
    \alpha_h(V_{E_k})   & = \frac{0.03(V_{E_k} + 45)}{1 - e^{-\frac{V_{E_k} + 45}{1.5}}},              & \beta_h(V_{E_k}) & = \frac{0.01(V_{E_k} + 45)}{e^{\frac{V_{E_k} + 45}{1.5}} - 1},  \\
    i_{\infty}(V_{E_k}) & = \frac{1 + b_k e^{\frac{V_{E_k} + 60}{2}}}{1 + e^{\frac{V_{E_k} + 60}{2}}}, & \tau_i           & = \max\left(10, \frac{30000\beta_i}{1 + \alpha_i}\right),       \\
    \alpha_i(V_{E_k})   & = e^{0.45(V_{E_k} + 66)},                                                    & \beta_i(V_{E_k}) & = e^{0.09(V_{E_k} + 66)},                                       \\
    n_{\infty}          & = \frac{1}{1 + \alpha_n},                                                    & \tau_n           & = \max\left(2, \frac{50\beta_n}{1 + \alpha_n}\right),           \\
    \alpha_n(V_{E_k})   & = e^{-0.11(V_{E_k} - 13)},                                                   & \beta_n(V_{E_k}) & = e^{-0.08(V_{E_k} - 13)},
\end{align*}
\begin{align*}
    r_{\infty,\text{HCN2}}(V_{E_k}) & = \frac{1}{1 + e^{\frac{V_{E_k} - V_{50k}}{10.5}}},                                         & \tau_{r,\text{HCN2}}(V_{E_k}) & = \frac{1}{e^{-14.59 - 0.086V_{E_k} + e^{-1.87 + 0.0701V_{E_k}}}},                          \\
    r_{\infty,\text{HCN1}}(V_{E_k}) & = \frac{1}{1 + e^{0.151(V_{E_k}-V_{50k})}},                                                 & \tau_{r,\text{HCN1}}(V_{E_k}) & = \frac{e^{0.033(V_{E_k} + 75)}}{0.011 \left( 1 + e^{0.0833(V_{E_k}+75)} \right)},          \\
    a_{\infty}                      & = \frac{1}{1 + \alpha_a},                                                                   & \tau_a                        & = \max \left( 0.1, \frac{c_k \beta_a}{1 + \alpha_a} \right),                                \\
    \alpha_a(V_{E_k})               & = e^{-0.038 \left( d_k + \frac{1}{1 + {\frac{e^V_{E_k} + 40}{5}}} \right) (V_{E_k} - e_k)}, & \beta_a(V_{E_k})              & = e^{-0.038 \left( f_k + \frac{1}{1 + {\frac{e^V_{E_k} + 40}{5}}} \right) (V_{E_k} - e_k)}, \\
    b_{\infty}(V_{E_k})             & = \frac{1}{1 + e^{0.11(V_{E_k} + 56)}},                                                     & \tau_b(V_{E_k})               & = \max \left( 2, 0.26(V_{E_k} + 50) \right).
\end{align*}
For the experiments, the following parameters were used:
\(E_{Na,E_k} = 55\) mV, \(E_{K,E_k} = -90\) mV, \(E_{h,E_k} = -30\) mV, \(E_{A,E_k} = -90\) mV.
The ionic conductances values (in mS/cm\(^2\)) and values of other parameters dependent on the compartment are shown in the additional table section below.\pagebreak

\section{Additional data}

\section{Additional figures}

\section{Additional tables}

\todo[inline]{Add additional tables here for all the celltypes and for the noise with all the parameters}

\begin{table}[htbp]
    \centering
    \caption[Compartment dependent parameters]{Compartment dependent parameters. The ionic conductances are in mS/cm\(^2\).}
    \begin{tabular}{l|cccccccccc}
        \hline
        \hline
               & \( g_{h} \) & \( g_{A} \) & \( g_{Na} \) & \( g_{K} \) & \( V_{50} \) & \( b \) & \( c \) & \( d \) & \( e \) & \( f \) \\
        \hline
        Bdend  & 0.1         & 48          & 32           & 10          & -82          & 1       & 4       & 1.5     & 11      & 0.825   \\
        Soma   & 0.1         & 48          & 32           & 10          & -82          & 0.8     & 4       & 1.5     & 11      & 0.825   \\
        Adend1 & 0.2         & 72          & 32           & 10          & -82          & 0.5     & 4       & 1.5     & 11      & 0.825   \\
        Adend2 & 0.4         & 120         & 32           & 10          & -90          & 0.5     & 2       & 1.8     & -1      & 0.7     \\
        Adend3 & 0.7         & 200         & 32           & 10          & -90          & 0.5     & 2       & 1.8     & -1      & 0.7     \\
        \hline
        \hline
    \end{tabular}
\end{table}\label{table:compartment_dependent_parameters}

\begin{table}[htbp]
    \centering
    \caption[Synaptic Parameters for the Connectivity Between Neurons in the Model]{Synaptic Parameters for the Connectivity Between Neurons in the Model: Pre- and postsynaptic receptor types are given for each cell type.
        The time constants \(\tau_1\) and \(\tau_2\) are in milliseconds.
        \(\tau_1\) is the rise time constant, the time it takes for synaptic conductance to increase from baseline to peak.
        \(\tau_2\) is the decay time constant, the time it takes for the conductance to decrease from peak to baseline.
        The conductance indicates the strength of the synaptic connection and its ability to conduct ionic current across the postsynaptic membrane.
        This influences the extent to which the synaptic input can depolarize the postsynaptic neuron and is in nanoSiemens (nS).}
    \begin{tabular}{lllccc}
        \hline
        Presynaptic & Postsynaptic & Receptor & \(\tau_1\) (ms) & \(\tau_2\) (ms) & Conductance (nS) \\
        \hline
        Pyramidal   & Pyramidal    & AMPA     & 0.05            & 5.3             & 0.02             \\
        Pyramidal   & Pyramidal    & NMDA     & 15              & 150             & 0.004            \\
        Pyramidal   & Basket       & AMPA     & 0.05            & 5.3             & 0.36             \\
        Pyramidal   & Basket       & NMDA     & 15              & 150             & 1.38             \\
        Pyramidal   & OLM          & AMPA     & 0.05            & 5.3             & 0.36             \\
        Pyramidal   & OLM          & NMDA     & 15              & 150             & 0.72             \\
        Basket      & Pyramidal    & GABA-A   & 0.07            & 9.1             & 0.72             \\
        Basket      & Basket       & GABA-A   & 0.07            & 9.1             & 4.5              \\
        Basket      & OLM          & GABA-A   & 0.07            & 9.1             & 0.0288           \\
        OLM         & Pyramidal    & GABA-A   & 0.2             & 20              & 72               \\
        MS          & Basket       & GABA-A   & 20              & 40              & 1.6              \\
        MS          & OLM          & GABA-A   & 20              & 40              & 1.6              \\
        \hline
    \end{tabular}\label{tab:synaptic_parameters_copy}
\end{table}

\begin{table}[htbp]
    \centering
    \caption[NetStim Parameters Summary]{NetStim Parameters Summary. The ``Number'' of spikes for each stimulus is calculated using the formula \((1e3 / \text{Interval}) \cdot h.tstop\), adjusting based on the specified interval for each stimulus.
        This results in a target number of spikes over the simulation period.}
    \begin{tabular}{lcccl}
        \hline
        \textbf{Stimulus} & \textbf{Interval (ms)} & \textbf{Noise} & \textbf{Weight (e-3)}                   & \textbf{Target} \\
        \hline
        Pyr 1             & 1                      & 1              & 0.05                                    & somaAMPAf       \\
        Pyr 2             & 1                      & 1              & \(0.05 \cdot \text{pyr\_noise\_scale}\) & Adend3AMPAf     \\
        Pyr 3             & 1                      & 1              & 0.012                                   & somaGABAf       \\
        Pyr 4             & 1                      & 1              & 0.012                                   & Adend3GABAf     \\
        Pyr 5             & 100                    & 1              & \(6.5 \cdot \text{pyr\_noise\_scale}\)  & Adend3NMDA      \\ \hline
    \end{tabular}
\end{table}\label{table:netstimparams_reduced}