\section*{Abstract}
Temporal lobe epilepsy (TLE) is a frequently occurring form of epilepsy, commonly
associated with the hippocampus, particularly its CA3 subfield, which is noted
for its hyperexcitability and crucial role in seizure generation. This study
employs a computational model of the CA3 subfield, developed within the NEURON
simulation environment, comprising 800 pyramidal cells, 200 basket cells, and 200
oriens-lacunosum moleculare (OLM) interneurons. To simulate genetic profiles
characteristic of epilepsy, network alterations were introduced to sodium and
potassium conductances across all cell types. The study examines the impact of
these alterations on neural oscillations within the theta-gamma frequency range
and their power. Additionally, it investigates the network's susceptibility to
depolarization blocks in basket cell populations resulting from increased external
noise. An in-depth exploration of population burst dynamics was also conducted.
Furthermore, the effects of strengthened recurrent connections in the basket cell
population were examined to assess the potential for rescuing the network from an
ictal state back to homeostatic baseline activity. The findings suggest that the
CA3 network is indeed hyperexcitable, with imbalances in sodium and potassium
channels that mirror genetic predispositions for epilepsy, leading to increased
firing rates and heightened susceptibility to epileptiform activity. Moreover, the
network demonstrated increased resilience to depolarization blocks through enhanced
soma-inhibition by recurrently connected basket cells, showing effects akin to
those of contemporary anti-epileptic drugs (AEDs) and other therapeutic interventions.\\

\noindent
\textbf{Keywords:} Temporal lobe epilepsy; CA3; hippocampus; computational neuroscience; NEURON\@;
depolarization block; ictal state; oscillations