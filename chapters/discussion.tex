\chapter{Discussion}

This study used a computer model of the CA3 subfield of the hippocampus originally developed by \textcite{neymotinKetamineDisruptsTheta2011}.
This model was used to study the effects of several different parameters that could initiate an epileptic state.

Initially, the implementation of the baseline model was verified to ensure that the model was functioning as expected.
Similar metrics, to the ones tested in the adaptation by \textcite{sanjayImpairedDendriticInhibition2015} were used to verify the model.
These metrics included population firing rates and theta-gamma oscillations (power and frequency).
The experiments by \textcite{sanjayImpairedDendriticInhibition2015} explored the effects of impaired dendritic inhibition in the CA3 network
as the main cause of epilepsy in the highly vulnerable brain region that is the CA3 subfield.

The results of our baseline model were consistent with the results of the \textcite{sanjayImpairedDendriticInhibition2015} study,
which made it possible to proceed with variations and additional experiments using the same CA3 model in NEURON simulator.

First, the network dynamics due to dysfunctional voltage-gated ion channels for sodium and potassium
were investigated in pyramidal, basket and O-LM cells populations within a CA3 hippocampal network.
These variations of channel dynamics (conductance) were simulated to resemble epileptic genetic profiles which could be related TLE patients.

In addition to that, the networks susceptibility to external noise was tested by varying the noise level in the network with similar channel dynamics.
Per trial of a condition, the amount of occurrences of depolarization blocks in the basket cell populations were counted and the average delay in the
loss in basket cell activity was calculated.

Lastly, it was explored if the network could be rescued from an epileptic state by gradually increasing the strength of soma inhibition by basket cells
by increasing the weight of recurrent connection within the basket cell population. Again, counting the amount of occurrences of depolarization blocks
in the basket cell populations.
\pagebreak

\section{Model assumptions and Observations}
In the original experiments \textcite{sanjayImpairedDendriticInhibition2015} the authors studied if impaired dendritic
inhibition could lead to epileptic activity in the network.
In their three experiment scenario's, they impaired the dendritic inhibition by O-LM cells on pyramidal cells (1), increased external noise to distal
dendrites of pyramidal cells (2), and modified the connection strength of all cell types in the network (3).
Their findings had no previous experimental validation and a linear relationship was assumed.

In this study, the experiments build upon the observations made by \textcite{sanjayImpairedDendriticInhibition2015} and further explored the network dynamics.
Simulations were done separately, with step-wise modifications to parameters such as conductance or connection weight.
This was necessary in order to be able to determine the extent in which these parameters could potentially lead to epileptic activity.

In all experiments Medial Septum inputs were kept constant at 150 ms intervals to both O-LM and basket cell populations
and simulations ran for 5 seconds in each trial for consistency.

\subsection{Sodium and Potassium variants}
In the first experiment, sodium and potassium conductance parameters were modified
in each cell type separately to identify which population had the most influence
on firing activity of the other cell types and on the theta-gamma oscillations.

\subsubsection{Pyramidal cells}
In the case of pyramidal cells, modifications to sodium in O-LM cells had the most
significant effect on their firing rate (figure~\ref{fig:sodium_potassium_firing_rates}, left). Similarly to directly
reducing dendritic inhibition via modification of connection weight as in scenario 1 of the \textcite{sanjayImpairedDendriticInhibition2015},
the pyramidal cells showed increased activity in their activity as (\(g_{\text{Na}}\)) for O-LM cells was reduced.
This also increased the firing rates of the other two cell types, as they receive inputs from the
pyramidal cells (figure~\ref{fig:model_design}). The large increase in basket cell firing rate directly influenced
the gamma component of the LFP (figure~\ref{fig:sodium_potassium_power}), while O-LM reduced the theta component.
These findings are consistent with the results of the \textcite{sanjayImpairedDendriticInhibition2015} study.

Out of all the cell types, pyramidal cells show the most linear relationship between the conductance of sodium
and potassium channels and their firing rate. Potassium had an oppositely proportional effect on the firing rate of
pyramidal cells compared to sodium. This result is somewhat unsurprising considering the role of potassium. In normal action potential
dynamics, voltage-dependent sodium current rapidly depolarizes the membrane via voltage-gated ion channels. As a response so too are
voltage-activated potassium channels activated that hyperpolarize the membrane potential. This effect inactivates sodium channels
as the membrane potential goes back to its polarized resting state, which in turn tunes down the firing frequency as observed.

\subsubsection{Basket cells}
In the basket cells a similar trend is visible, although the firing rate shift is much more pronounced for both modified ions.
Again, this is a sensible result, considering the fast acting GABAergic inhibitory role as an interneuron~\parencite{wangGammaOscillationSynaptic1996}.
In addition, basket cells contain extensive axonal arborization that allow them to form many connections with pyramidal cells and
to other interneurons~\parencite{tukkerDistinctDendriticArborization2013}. Especially, their many connections to cells of their own type,
might result in highly synchronized firing patterns that were observed in the population activity (figure~\ref{fig:scatterplot_20_con_olm_pyr}).
This effect is also expressed in the model, through the many-to-one connection design of the basket cells (Table~\ref{tab:bwb_gaba_connections}).

\subsubsection{O-LM cells}
O-LM cells had again somewhat linear, but relative small impact on the firing rates.
Potentially, these effects are dependent on the morphology of the O-LM cells.
O-LM cells primarily target the distal dendrites of pyramidal cells in the stratum lacunosum moleculare.
This targeting influences their firing properties, as the distal dendritic locations typically receive inputs that are less intense or
less frequent compared to the somatic or proximal dendritic inputs received by basket cells. Additionally, O-LM cells have less extensive
local axonal arborizations compared to basket cells, which limits their range of influence and the synaptic inputs they receive~\parencite{saragaActiveDendritesSpike2003}.

The primary role of O-LM cells is to modulate the input to the hippocampus from the entorhinal cortex,
affecting the integration of cortical information~\parencite{leaoOLMInterneuronsDifferentially2012}. This modulation often requires precise,
but less frequent, inhibition compared to the broad, fast inhibition exerted by basket cells on the pyramidal cell bodies.
Therefore, their activity is more phasic or conditional, dependent on specific synaptic events that do not necessitate high-frequency firing.

\subsubsection{Theta-gamma oscillations}
Interestingly, theta remains relatively stable across all cell types, while gamma oscillations are more affected by the conductance modifications.
Like in the original studies using the same model by \textcite{sanjayImpairedDendriticInhibition2015,neymotinKetamineDisruptsTheta2011},
theta is resilient to changes in the network. This resilience is mostly due to the strong pacing which is provided to the interneurons by the MS\@.
For sodium, while the theta component of the LFP appears to change a lot per condition, they are a factor 10 times smaller than does observed in the gamma component.
With potassium modifications however, only the pyramidal cells seem to resemble a linear relationship with the theta power compared to the other cell types.

Potassium conductance influences the speed at which the membrane potential returns to its resting state.
In the case of the the studied network, increased potassium conductance potentially causes faster repolarization of action potentials.
This can shorten the duration of individual spikes and reduce the overall excitability of pyramidal cells, leading to a lower firing rate.
However, this rapid repolarization can also allow neurons to return more quickly to a state where they can fire again, potentially aiding in
synchronization~\parencite{mysinMechanismsFunctionsRole2022}.
Alternatively, The quick recovery of neurons to their resting or near-threshold state can facilitate better phase-locking among neurons within
the network~\parencite{leungPhasicModulationHippocampal2020a}. This synchronization is crucial for enhancing the coherence and power of network
oscillations like theta.

GABAergic interneurons, such as the basket and O-LM cells play a significant role in shaping the response of the network.
As pyramidal cells fire less frequently or less synchronously due to increased potassium conductance,
basket and O-LM interneurons effectively increase their control over the timing of pyramidal cell output~\parencite{unalSpatiotemporalSpecializationGABAergic2018}.
This can help enforce a more regimented and synchronized firing pattern in line with the theta rhythm.

Gamma oscillations, on the other hand, are more sensitive to changes in the network. Especially in the case for O-LM modifications of at least -30 \% (\(g_{\text{Na}}\))
(figure~\ref{fig:sodium_potassium_power}, bottom left). As the O-LM cells disinhibit the pyramidal cells, the basket cells receive more excitatory input which
increases the gamma component of the LFP\@. The gamma power promptly returns to baseline levels when the O-LM cells are restored to their original conductance.
Again showcasing, the high sensitivity of basket cells to changes in the network.

Classically it is thought that fast gamma oscillations are generated by the recurrent excitation between principle (pyramidal) neurons in
generation of the theta-modulated gamma rhythm~\parencite{wangGammaOscillationSynaptic1996}. However, synchronization of basket cells and
desynchronization of the other cell types in the more extreme experimental conditions suggests that the synaptic connection might be pivotal
in the observed response (example of such effects in are seen in figures~\ref{fig:scatterplot_20_con_olm_pyr} and~\ref{fig:scatterplot_0_con_olm_pyr}).

Early research has already shown that the type of synaptic input on AMPA type receptors can actually desynchronize, rather than synchronize
a network of coupled neurons like in the model of this study~\parencite{vanvreeswijkWhenInhibitionNot1994,khazipovSynchronizationGABAergicInterneuronal1997}.
Thus, the oscillatory coherence in the network might be more dependent on rhythmic inhibition by fast-spiking interneurons
like basket cells~\parencite{lyttonSimulationsCorticalPyramidal1991}.

More modern investigations seem to support the idea that contrasting neurotransmitter release at the synapse, like glutamate/GABA or NMDA/AMPA co-transmission
can regulate input-output dynamics in hippocampal circuits, rather than just membrane
conductance dynamics~\parencite{ajibolaHypothalamicGlutamateGABA2021,micheliMechanisticModelNMDA2021}.

\subsection{External noise variants}
Throughout this experiment, O-LM-pyramidal connection strength was kept at 10 \% of the baseline value.
In the \textcite{sanjayImpairedDendriticInhibition2015} paper, the authors increased the noise through distal dendrites of pyramidal cells up to 15 times,
where they noticed that the basket cell population entered a depolarization block. Similarly, in this study the presence of a depolarization block is the main factor
that indicates the pathological state in the CA3 network. In this state the driving force is the increased pyramidal cell activity which lacks dendritic inhibition.
In this situation, the LFP signal of the network showed an ictal-tonic pattern, as did the
population (figure~\ref{fig:depolarization_block_basket_cells} and~\ref{fig:scatterplot_1_con_olm_pyr_ext_noise_20x}).

\subsubsection{Percentage matrices}
Comparing all the noise levels, it is readily apparent that in higher sodium conductance conditions, the network is much more excitable.
As noise increases, elevated sodium rapidly induces an epileptic state 100 \% of the time even though the jump in external noise goes from 16 to 17 times the
baseline (0.80 and 0.85 noise levels, figure~\ref{fig:dpb_percentage_matrices}). Likewise, lower sodium levels are far less susceptible to hyperexcitability.
At around -50 \% (\(g_{\text{Na}}\)), the input from pyramidal cells is barely strong enough to induce a depolarization block even at 20 times baseline noise.
In combination with elevated potassium, the network showed the most severe susceptibility for epileptic activity.
From the noise matrices it becomes apparent that the network is quite sensitive to small changes in noise level, where increments of 0.05 in noise levels
induced significantly more depolarization blocks generally for each condition.

\subsubsection{Delay matrices}
The onset of the depolarization block changes with the noise level and the conductance of sodium and potassium.
In the conditions with more DPBs, the average delay becomes significantly shorter and with less variance (figure~\ref{fig:dpb_delay_matrices}).
The original~\textcite{sanjayImpairedDendriticInhibition2015} study states in the discussion that the shorter delay in the depolarization block
that they perceived could be due to difference in synaptic plasticity that could include axonal or dendritic sprouting (scenario 3: effect of changes in connectivity at all
synapses in the network). However, synapses (except the O-LM to pyr connection) were unchanged when measuring the delay of the DPB in this experiment.
Yet, enhanced neural activity presented here showed similar results to that of a potentiated network that has been observed in other models with modified synaptic strength~\parencite{leitePlasticitySynapticStrength2005}.
Key is that the faster generation of epileptic activity, while sometimes result of long-term potentiation, usually occurs in various network circuitry throughout the brain~\parencite{cookePlasticityHumanCentral2006}.

\subsubsection{Burst Analysis}
A deeper investigation into the burst activity of the population was performed for basket and pyramidal cells.
Considering that pyramidal activity is the apparent driving force behind the depolarization block, the exact reason of initiation was unclear.
Therefore, a closer look at the burst timings and peak sizes reveals that beyond the baseline there are some consistent patterns (figure~\ref{fig:burst_detection}).

During the inter-ictal state the basket cells burst came before the pyramidal cells, whereas all other noise conditions did not show this burst pattern.
In the pre-ictal state the basket cell burst seemed to shift in time, from before, to after the pyramidal cells burst.
Thus, it is not always the case that the pyramidal input comes slightly before the basket cell burst, when the population might be in the refractory period and become locked in a depolarization block
due to the constant excessive excitatory drive.

The ictal pyramidal activity did not show much variance from the baseline, in shape and in intensity.
Throughout the trials the intensity of the population burst did not vary as is seen by the very small deviation, but the size of the generated noise also never did.
This suggested that the network is quite sensitive to noise. Moreover, the amount of noise did not necessarily determine the characteristics of the bursts themselves.

Pyramidal cells are known to depend on the interaction between glutamatergic and GABAergic receptor activity, a powerful mechanism
of intra-burst spike frequency modulation~\parencite{dzhalaExcitatoryActionsEndogenously2003}. Although the basket cells do not fire at all after a DPB, the remaining O-LM influence (10 \%)
on pyramidal cells is perhaps enough to keep the pyramidal burst consistent.

The data suggests that the all-or-none bursting of pyramidal or basket cells is not necessarily dependent on precise timing.
Rather, the ability for the population to recover from strong synchronized input appeared to be an issue especially for basket cells.

This non-dependence on timing has been explored for the hippocampus by \textcite{menendezdelapridaThresholdBehaviorInitiation2006}, where
they identified that populations actually go through three phases of firing periodically. These include: a recovery phase (from the previous burst), a plateau period of fluctuating activity,
and a buildup were the firing rate accelerates just before the burst. These phases are also recognizable in the shown bursts of pyramidal and basket cells in figure~\ref{fig:burst_detection}.

However, it remains unclear if the timing of the proposed epileptic discharge (near the onset of a DPB) of pyramidal cells actually matters in the initiation of the depolarization block in basket cells.
There are hints in the literature that sustained inter-ictal and ictal-like activity might even co-occur within the same network, provided the dysfunction of neural populations
is beyond a certain excitability threshold~\parencite{dzhalaTransitionInterictalIctal2003}.

\subsection{Recurrent connection strength variants}
A further investigation in the depolarization block phenomenon was done in the last experiment.
Here it was explored if increasing input among basket cells via increasingly stronger recurrent connection strength
could overpower the massive external drive from the pyramidal cells and rescue the network from entering into the epileptic state.
Similarly to the noise experiment, in enhanced excitatory conditions with high sodium and potassium conductance it was significantly less effective to reduce
the amount of DPBs via enhanced recurrent inhibition.

The twenty times external input from the entorhinal cortex resembles heavy visual or auditory stimuli that possibly triggers
epilepsy in a CA3 network. Nevertheless, the results show that even in the situation of disinhibition by O-LM cells (10 \% connection strength),
the network can be strengthened to almost completely avoid depolarization blocks in basket cells (figure~\ref{fig:rc_dpb_percentage_matrices}).
This could suggest that patients suffering from TLE could indeed be helped via anti-epileptic drugs (AEDs) such as benzodiazepines or barbiturates.
Especially those that specifically stimulate GABAergic interneurons, such as basket cells that can dampen the firing of pyramidal cells of the hippocampus.

\section{Implications of the findings}
The results of this study highlight the critical role of ion channel dynamics, particularly sodium and potassium, in the generation of epileptic activity in the CA3 network.
Modifications in these channels simulate conditions akin to those of specific genetic profiles that are associated with TLE\@.
Furthermore, the network's response to small variations in elevated external noise levels in such conditions show that the CA3 subfield is indeed exceptionally
excitable and susceptible to enter an epileptic state. In addition, the role of interneurons in network stability became apparent as
even in a case of impaired dendritic inhibition by O-LM cells, epileptic networks can be rescued by influence of other interneurons like basket cells.
Therefore, inducing such effects could be the focal point of medicinal intervention.

These results thus imply that manipulating interneuron activity, particularly through enhancing soma inhibition provides a theoretical bases for therapeutic strategies.
Simulating the various scenarios of this study and observing the outcomes, has shown that this computer model could serve as a predictive tool for understanding how certain genetic
or environmental changes might precipitate epilepsy. This could be invaluable for developing personalized medicine approaches for TLE management.

The simulation's ability to reproduce and explore effects of synaptic plasticity via connection strength, offers a window into how long-term changes
at synapses might contribute to epilepsy. This aligns with contemporary research suggesting that alterations in synaptic strength and
plasticity play a significant role in the disease's progression.

\section{Limitations of the research}
As previously mentioned, there are many models available on ModelDB that explore the CA3 network in the hippocampus
\href{http://senselab.med.yale.edu/modeldb}{(\url{http://senselab.med.yale.edu/modeldb})}.

The~\textcite{neymotinKetamineDisruptsTheta2011} model adapted by~\textcite{sanjayImpairedDendriticInhibition2015} in particular was chosen
because of the relative high degree of complexity and the presence of homeostatic mechanisms, such as theta-modulated gamma oscillations
as a baseline. The individual neuron types also contained many biophysical properties displayed in models and experiments of normal physiology.
However, the model is still a simplification of the actual biological system and is lacking in terms of the full complexity of the CA3 network.

Also, the specific focus on the CA3 subfield and its particular cell types limits the generalizability of the findings to other regions of the hippocampus,
even other subfields like CA1 or the dentate gyrus. Such that the unique properties and connections in the CA3 subfield do not
reflect the broader dynamics involved in TLE\@.

Moreover, the simulation's results are heavily dependent on the initial assumptions and parameters set by the original researchers.
The linear-relationship that was assumed in terms of synaptic interactions may not hold under more realistic conditions.
The simulations also have a limited scope in terms of the number of trials, which was a decision made based on the computational resources available.
Yet, there are indications that due to random seeding (figure~\ref{tab:validation_results}), more trials would possibly yield more consistent results.

Other earlier models have studied the influence of specific currents, conductances and synaptic connections in relationship with the generation
of epileptic activity. Of which, CA3 has been investigated fewer times still, partially due to the heterogenous nature of this region.

Single cell conductance-based models such as the one designed by~\textcite{cressmanInfluenceSodiumPotassium2009}, have already shown that role of ion concentration dynamics
even in reduced models are amenable. They are both qualitatively and quantitatively similar to full epilepsy network models such the one from this study. They also
adhere to the Hodgkin-Huxley formalism which makes comparison between models easier.

Integrating aspects of single cell models with more parameters remains difficult however, as they increase the complexity of bifurcation analysis in neural networks.
Meanwhile the identification ictal state transitions also becomes more difficult as the number of parameters increases.

\section{Suggestions for future research}
Like in the original model, only two main inhibitory mechanisms are integrated. Namely somatic and dendritic through basket and O-LM cells respectively.
However, in vivo, the CA3 network contains many more interneuron types that could be included in the model.

A goal of this study was to explore if certain genetic profiles could be translated into model dynamics.
Genotype-phenotype relationships are complex and often involve multiple genes and environmental factors~\parencite{adiga*TherapeuticsEpilepsyReview2023}.
Therefore, future endeavors could focus on incorporating data from sequenced genomes of TLE patients linked to channelopathies.
This could provide a more accurate representation of the genetic basis of epilepsy and how it manifests in the CA3 network,
provided there is data on how these mutations manifest in channel dynamics.

In addition, managing epilepsy is not only about preventing seizures, but also about improving the quality of life for patients.
The key to that remains providing adequate medication with AEDs tailored to the individual form of epilepsy in the patient.

Therapeutic approaches aimed at reinstating the disrupted balance between excitatory and inhibitory processes hold significant potential for mitigating pathological brain activities.
The integration of computational research with forthcoming experimental studies presents a robust avenue for enhancing our current understanding and formulating effective treatments for neurodegenerative diseases.
This multidisciplinary strategy is expected to yield substantial advancements in the field as technology and the capacity for realistic simulations improves.