\chapter{Discussion}

This chapter holds the discussion of the results and the implications of the
findings. It also includes the limitations of the research and suggestions for
future research.

\section{Discussion of the results}
This study used a computer model of the CA3 subfield of the hippocampus originally developed by \textcite{neymotinKetamineDisruptsTheta2011}.
This model was used to study the effects of several different parameters that could initiate an epileptic state.

Initially, the implementation of the baseline model was verified to ensure that the model was functioning as expected.
Similar metrics, to the ones tested in the adaptation by \textcite{sanjayImpairedDendriticInhibition2015} were used to verify the model.
These metrics included population firing rates and theta-gamma oscillations (power and frequency).
The experiments by \textcite{sanjayImpairedDendriticInhibition2015} explored the effects of impaired dendritic inhibition in the CA3 network
as the main cause of epilepsy in the highly vulnerable brain region that is the CA3 subfield.

The results of our baseline model were consistent with the results of the \textcite{sanjayImpairedDendriticInhibition2015} study,
which made it possible to proceed with variations and additional experiments using the same CA3 model in NEURON simulator.

First, the network dynamics due to dysfunctional voltage-gated ion channels for sodium and potassium 
were investigated in pyramidal, basket and O-LM cells populations within a CA3 hippocampal network.
These variations of channel dynamics (conductance) were simulated to resemble epileptic genetic profiles which could be related TLE patients.

In addition to that, the networks susceptibility to external noise was tested by varying the noise level in the network with similar channel dynamics.
Per trial of a condition, the amount of occurrences of depolarization blocks in the basket cell populations were counted and the average delay in the 
loss in basket cell activity was calculated.

Lastly, it was explored if the network could be rescued from an epileptic state by gradually increasing the strength of soma inhibition by basket cells 
by increasing the weight of recurrent connection within the basket cell population. Again, counting the amount of occurrences of depolarization blocks 
in the basket cell populations.

\section{Implications of the findings}

\section{Limitations of the research}

\section{Suggestions for future research}