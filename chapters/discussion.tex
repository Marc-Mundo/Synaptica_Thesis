\chapter{Discussion}

This chapter holds the discussion of the results and the implications of the
findings. It also includes the limitations of the research and suggestions for
future research.

\section{Discussion of the results}
This study used a computer model of the CA3 subfield of the hippocampus originally developed by \textcite{neymotinKetamineDisruptsTheta2011}.
This model was used to study the effects of several different parameters that could initiate an epileptic state.

Initially, the implementation of the baseline model was verified to ensure that the model was functioning as expected.
Similar metrics, to the ones tested in the adaptation by \textcite{sanjayImpairedDendriticInhibition2015} were used to verify the model.
These metrics included population firing rates and theta-gamma oscillations (power and frequency).
The experiments by \textcite{sanjayImpairedDendriticInhibition2015} explored the effects of impaired dendritic inhibition in the CA3 network
as the main cause of epilepsy in the highly vulnerable brain region that is the CA3 subfield.

The results of our baseline model were consistent with the results of the \textcite{sanjayImpairedDendriticInhibition2015} study,
which made it possible to proceed with variations and additional experiments using the same CA3 model in NEURON simulator.

First, the network dynamics due to dysfunctional voltage-gated ion channels for sodium and potassium 
were investigated in pyramidal, basket and O-LM cells populations within a CA3 hippocampal network.
These variations of channel dynamics (conductance) were simulated to resemble epileptic genetic profiles which could be related TLE patients.

In addition to that, the networks susceptibility to external noise was tested by varying the noise level in the network with similar channel dynamics.
Per trial of a condition, the amount of occurrences of depolarization blocks in the basket cell populations were counted and the average delay in the 
loss in basket cell activity was calculated.

Lastly, it was explored if the network could be rescued from an epileptic state by gradually increasing the strength of soma inhibition by basket cells 
by increasing the weight of recurrent connection within the basket cell population. Again, counting the amount of occurrences of depolarization blocks 
in the basket cell populations.

\section{Model assumptions and Observations}
In the original experiments \textcite{sanjayImpairedDendriticInhibition2015} the authors studied if impaired dendritic 
inhibition could lead to epileptic activity in the network.
In their three experiment scenario's, they impaired the dendritic inhibition by O-LM cells on pyramidal cells (1), increased external noise to distal 
dendrites of pyramidal cells (2), and modified the connection strength of all cell types in the network (3). 
Their findings had no previous experimental validation and a linear relationship was assumed.

In this study, the experiments build upon the observations made by \textcite{sanjayImpairedDendriticInhibition2015} and further explored the network dynamics.
Simulations were done separately, with step-wise modifications to parameters such as conductance or connection weight.
This was necessary in order to be able to determine the extent in which these parameters could potentially lead to epileptic activity.

In all experiments Medial Septum inputs were kept constant at 150 ms intervals to both O-LM and basket cell populations 
and simulations ran for 5 seconds in each trial for consistency.

\subsection{Experiment 1: Sodium and Potassium variants}
In the first experiment, sodium and potassium conductance parameters were modified
in each cell type separately to identify which population had the most influence
on firing activity of the other cell types and on the theta-gamma oscillations.

\subsubsection{Pyramidal cells}
In the case of pyramidal cells, modifications to sodium in O-LM cells had the most 
significant effect on their firing rate (figure~\ref{fig:sodium_potassium_firing_rates}, left). Similarly to directly 
reducing dendritic inhibition via modification of connection weight as in scenario 1 of the \textcite{sanjayImpairedDendriticInhibition2015},
the pyramidal cells showed increased activity in their activity as (\(g_{\text{Na}}\)) for O-LM cells was reduced. 
This also increased the firing rates of the other two cell types, as they receive inputs from the 
pyramidal cells (figure~\ref{fig:model_design}). The large increase in basket cell firing rate directly influenced
the gamma component of the LFP (figure~\ref{fig:sodium_potassium_power}), while O-LM reduced the theta component.
These findings are consistent with the results of the \textcite{sanjayImpairedDendriticInhibition2015} study.

Out of all the cell types, pyramidal cells show the most linear relationship between the conductance of sodium 
and potassium channels and their firing rate. Potassium had an oppositely proportional effect on the firing rate of 
pyramidal cells compared to sodium. This result is somewhat unsurprising considering the role of potassium. In normal action potential
dynamics, voltage-dependent sodium current rapidly depolarizes the membrane via voltage-gated ion channels. As a response so too are
voltage-activated potassium channels activated that hyperpolarize the membrane potential. This effect inactivates sodium channels
as the membrane potential goes back to its polarized resting state, which in turn tunes down the firing frequency as observed. 

\subsubsection{Basket cells}
In the basket cells a similar trend is visible, although the firing rate shift is much more pronounced for both modified ions.
Again, this is a sensible result, considering the fast acting GABAergic inhibitory role as an interneuron~\parencite{wangGammaOscillationSynaptic1996}. 
In addition, basket cells contain extensive axonal arborization that allow them to form many connections with pyramidal cells and 
to other interneurons~\parencite{tukkerDistinctDendriticArborization2013}. Especially, their many connections to cells of their own type,
might result in highly synchronized firing patterns that were observed in the population activity (Figure~\ref{fig:scatterplot_20_con_olm_pyr}).
This effect is also expressed in the model, through the many-to-one connection design of the basket cells (Table~\ref{tab:bwb_gaba_connections}).

\subsubsection{O-LM cells}
O-LM cells had again somewhat linear, but relative small impact on the firing rates.
Potentially, these effects are dependent on the morphology of the O-LM cells.
O-LM cells primarily target the distal dendrites of pyramidal cells in the stratum lacunosum moleculare. 
This targeting influences their firing properties, as the distal dendritic locations typically receive inputs that are less intense or 
less frequent compared to the somatic or proximal dendritic inputs received by basket cells. Additionally, O-LM cells have less extensive 
local axonal arborizations compared to basket cells, which limits their range of influence and the synaptic inputs they receive~\parencite{saragaActiveDendritesSpike2003}.

The primary role of O-LM cells is to modulate the input to the hippocampus from the entorhinal cortex, 
affecting the integration of cortical information~\parencite{leaoOLMInterneuronsDifferentially2012}. This modulation often requires precise, 
but less frequent, inhibition compared to the broad, fast inhibition exerted by basket cells on the pyramidal cell bodies. 
Therefore, their activity is more phasic or conditional, dependent on specific synaptic events that do not necessitate high-frequency firing.

\subsubsection{Theta-gamma oscillations}
Interestingly, theta remains relatively stable across all cell types, while gamma oscillations are more affected by the conductance modifications.
Like in the original studies using the same model by \textcite{sanjayImpairedDendriticInhibition2015,neymotinKetamineDisruptsTheta2011}, 
theta is resilient to changes in the network. This resilience is mostly due to the strong pacing which is provided to the interneurons by the MS\@.

Gamma oscillations, on the other hand, are more sensitive to changes in the network. Especially in the case for O-LM modifications of at least -30 \% (\(g_{\text{Na}}\))
(figure~\ref{fig:sodium_potassium_power}, bottom left). As the O-LM cells disinhibit the pyramidal cells, the basket cells receive more excitatory input which 
increases the gamma component of the LFP\@. The gamma power promptly returns to baseline levels when the O-LM cells are restored to their original conductance.
Again showcasing, the high sensitivity of basket cells to changes in the network. 

Classically it is thought that fast gamma oscillations are generated by the recurrent excitation between principle (pyramidal) neurons in 
generation of the theta-modulated gamma rhythm~\parencite{wangGammaOscillationSynaptic1996}. However, synchronization of basket cells and 
desynchronization of the other cell types in the more extreme experimental conditions suggests that the synaptic connection might be pivotal 
in the observed response (example of such effects in are seen in figures~\ref{fig:scatterplot_20_con_olm_pyr}, and~\ref{fig:scatterplot_0_con_olm_pyr}).

Early research has already shown that the type of synaptic input on AMPA type receptors can actually desynchronize, rather than synchronize
a network of coupled neurons like in the model of this study~\parencite{vanvreeswijkWhenInhibitionNot1994,khazipovSynchronizationGABAergicInterneuronal1997}. 
Thus, the oscillatory coherence in the network might be more dependent on rhythmic inhibition by fast-spiking interneurons like basket cells. 

More modern investigations seem to support the idea that contrasting neurotransmitter release at the synapse, like glutamate/GABA or NMDA/AMPA co-transmission 
can regulate input-output dynamics in hippocampal circuits, rather than just membrane 
conductance dynamics~\parencite{ajibolaHypothalamicGlutamateGABA2021,micheliMechanisticModelNMDA2021}.

\subsection{Experiment 2: External noise}


\section{Implications of the findings}

\section{Limitations of the research}

\section{Suggestions for future research}