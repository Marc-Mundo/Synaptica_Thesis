\chapter{Results}

\todo[inline]{fix the captions for the figures, see the article for the correct captions.}

\section{Results of the Baseline activity}
In the initial experiment, the baseline activity of the CA3 network was
observed from the \textit{original model} by
\textcite{sanjayImpairedDendriticInhibition2015}. The network was simulated for
5000 ms and showed synchronous activity throughout all three
populations (pyr, BC and OLM). Basket cells showed a higher firing rate
compared to the pyramidal cells and O-LM cells. The basket cells also swapped
between states of synchrony and asynchrony which was not observed in the other
two populations. The O-LM cells showed the lowest firing rate compared to the
pyramidal cells and basket cells. The baseline activity of the CA3 network is
shown in figure~\ref{fig:baseline_activity}.

The average firing rates of the populations were 2.36 ± 0.024 Hz for pyramidal cells, 
16.05 ± 0.15 Hz for basket cells, and 0.96 ± 0.027 Hz for O-LM interneurons, 
similar to observed results from \textcite{neymotinKetamineDisruptsTheta2011} which used the same model on which the Sanjay model is based upon.

Just as reported in the original article, the network produces theta-modulated gamma oscillations 
within the local field potential (LFP). These oscillations were influenced by signals from the Medial Septum (MS). 
The gamma oscillations emerged from the inhibitory connections between basket cells that inhibit somas of pyramidal and O-LM cells, 
as well as interactions among basket cells themselves. Conversely, theta oscillations were the result of interactions between 
pyramidal cells and O-LM cells that inhibit dendrites. The network achieved a consistent theta frequency of 6.7 Hz due to periodic inputs 
from the MS to both O-LM and basket cells every 150 ms. The frequency of the gamma component within the LFP was approximately 33 Hz. 
Despite receiving similar MS inputs as O-LM cells, the impact on basket cells was significantly reduced because of their mutual interactions 
and the enhanced influence from pyramidal cells.

\begin{figure}[htbp]
    \centering
    \includegraphics[width=1.0\textwidth]{Network_spike_activity_OLM_baseline.png}
    \caption[Baseline activity of the CA3 network]{Baseline activity of the CA3 network.}\label{fig:baseline_activity}
    \begin{minipage}{0.9\textwidth}
        The above figure shows the baseline activity of the CA3 network. The network was simulated for 5000 ms. 
        The spike activity in time of the Pyr cells, BC cells, and OLM cells are shown based on the Neuron ID\@. 
        ID 0--799 = Pyramidal (blue), 800--999 = BC (green), 1000--1200 = OLM (red). 
        The x-axis represents the time in ms and the y-axis represents the neuron ID\@.
    \end{minipage}
\end{figure}
\pagebreak
\section{Results of the Model validation}
To test whether our implementation of the CA3 network was able to replicate
more elaborate results, results from figure 6 of the original
\textcite{sanjayImpairedDendriticInhibition2015} article were replicated in
figure~\ref{fig:validation_firing_rates},~\ref{fig:validation_frequencies} and
~\ref{fig:validation_power}.

Like in the original experiment, O-LM-pyramidal connectivity was decreased in decrements of 0.1 and reduced dendritic inhibition.
Simultaneously, external noise fed to pyramidal cells via AMPA and NMDA at the synaptic level was increased in increments of 0.1. 
When O-LM-pyramidal connectivity was decreased to a range of 20--10 \%, desynchronization was observed among basket cells (figure~\ref{fig:scatterplot_20_con_olm_pyr}).
Complete desynchronization was observed when the O-LM-pyramidal connectivity was reduced to 0 \% (figure~\ref{fig:scatterplot_0_con_olm_pyr}).
Pyramidal to O-LM connectivity was was unchanged, thus these cells showed sustained synchronous activity.

\begin{figure}[htbp]
    \centering
    \includegraphics[width=1.0\textwidth]{Olm_pyr_20_con_scatter.png}
    \caption[20 \% OLM-Pyr connection scatter plot]{Scatter plot of the network activity at 20 \% OLM-Pyr connection}\label{fig:scatterplot_20_con_olm_pyr}
    \begin{minipage}{0.9\textwidth}
        The above figure shows the network activity at 20 \% OLM-Pyr connection with significant asynchrony amongst the basket cells.
        The network was simulated for 5000 ms. 
        The scatter plot shows the spike activity of the Pyr cells (blue), BC cells (green), and OLM cells (red). 
        The x-axis represents the time in ms and the y-axis represents the neuron ID\@. 
    \end{minipage}
\end{figure}

\begin{figure}[htbp]
    \centering
    \includegraphics[width=1.0\textwidth]{Olm_pyr_no_connection_scatter.png}
    \caption[0 \% OLM-Pyr connection scatter plot]{Scatter plot of the network activity at 0 \% OLM-Pyr connection}\label{fig:scatterplot_0_con_olm_pyr}
    \begin{minipage}{0.9\textwidth}
        The above figure shows the network activity at 0 \% OLM-Pyr connection with complete asynchrony amongst the basket cells.
        The network was simulated for 5000 ms. 
        The scatter plot shows the spike activity of the Pyr cells (blue), BC cells (green), and OLM cells (red). 
        The x-axis represents the time in ms and the y-axis represents the neuron ID\@. 
    \end{minipage}
\end{figure}

The results show that the model was able to
replicate the results of the original article with slightly lower firing rates,
theta-gamma frequencies and power, which can be seen in
table~\ref{tab:validation_results}. The original article results are visible in
the methods section in table~\ref{tab:original_validation_results} for comparison.

For the firing rates in figure~\ref{fig:validation_firing_rates},
there was a notable increase in the firing rates of all neuron types as dendritic inhibition was 
decreased while external noise was simultaneously increased.
The individual cell firing frequencies showed a near linear increase throughout in both pyramidal and O-LM cells, 
being most pronounced in the basket cell population.
The basket cell population also showed the most variance in the standard deviation at the most extreme 
condition from the baseline (0.0x OLM-Pyr weight and 2.0x external weight).

The dominant frequencies in the network activity, as shown in figure~\ref{fig:validation_frequencies},
showed that the theta frequency remained constant, while the gamma frequency only increased as dendritic inhibition was severely decreased and external noise increased.
The theta frequency remained constant at 6.2 Hz, slightly lower than the original model's 6.7 Hz, due to the strong pacing from the MS at this frequency.

The power of the theta and gamma oscillations in the network, as shown in figure~\ref{fig:validation_power},
showed that the power of theta oscillations decreased, while the gamma power increased as dendritic inhibition was reduced.
This shift in power distribution reflects changes in the balance of network excitability and inhibition, potentially leading to epileptic activity.
The theta power reduced to 5.35 mV\textsuperscript{2} Hz\textsuperscript{-1} to 0 mV\textsuperscript{2} Hz\textsuperscript{-1}.
The gamma power increased significantly from at 20 to 10 \% O-LM-pyramidal connection. The gamma power increased from 0.93 mV\textsuperscript{2} Hz\textsuperscript{-1} 
at baseline to 5.82 mV\textsuperscript{2} Hz\textsuperscript{-1} before dropping down to 1.87 mV\textsuperscript{2} Hz\textsuperscript{-1} in the last condition.
The gamma power was mostly due to basket cell activity, which were much more tightly synchronized than the other cell types.

The trends in the original figures were similar in figure~\ref{fig:validation_original_results} to the ones of this section.
Therefore, it was assumed that the model was correctly implemented and the results were valid.

\begin{figure}[htbp]
    \centering
    \includegraphics[width=1.0\textwidth]{Sanjay_validation_firing_rates.png}
    \caption[Validation of the firing rates]{Validation of the firing rates.}\label{fig:validation_firing_rates}
    \begin{minipage}{0.9\textwidth}
        The above figure shows the firing rates of the Pyr cells, BC cells, and O-LM cells when dendritic inhibition is decreased and external noise is increased.
        The firing rates were calculated from the spike activity of the cells in each population for the duration of the simulation (5000 ms).
        The double x-axis represents both decrement in the weight of dendritic inhibition on pyramidal cells by OLM interneurons,
        while simultaneously increasing external noise stimulation to pyramidal cells.
        External noise levels rise and inhibition decreases by increments of 0.1, each representing a 10\% change relative to the baseline.
        The y-axis represents the firing rate in Hz.
        The firing rates are per cell type: Pyr (blue), Basket (green) and OLM (red).
        The error bars represent the standard deviation of the firing rates.
    \end{minipage}
\end{figure}

\begin{figure}[htbp]
    \centering
    \includegraphics[width=1.0\textwidth]{Sanjay_validation_frequencies.png}
    \caption[Validation of the firing rates]{Validation of dominant frequencies.}\label{fig:validation_frequencies}
    \begin{minipage}{0.9\textwidth}
        The above figure shows the dominant theta-gamma frequencies in the network activity when dendritic inhibition is decreased and external noise is increased.
        The double x-axis represents both decrement in the weight of dendritic inhibition on pyramidal cells by O-LM interneurons,
        while simultaneously increasing external noise stimulation to pyramidal cells.
        External noise levels rise and inhibition decreases by increments of 0.1, each representing a 10\% change relative to the baseline.
        The y-axis represents the dominant frequency in Hz for both theta (3--12 Hz, blue) and gamma (30--80 Hz, orange) oscillatory bands.
    \end{minipage}
\end{figure}

\begin{figure}[htbp]
    \centering
    \includegraphics[width=1.0\textwidth]{Sanjay_validation_power.png}
    \caption[Validation of the firing rates]{Validation of Theta-Gamma power.}\label{fig:validation_power}
    \begin{minipage}{0.9\textwidth}
        The above figure shows the power of the theta and gamma oscillations in the network when dendritic inhibition is decreased and external noise is increased.
        The double x-axis represents both decrement in the weight of dendritic inhibition on pyramidal cells by O-LM interneurons,
        while simultaneously increasing external noise stimulation to pyramidal cells.
        External noise levels rise and inhibition decreases by increments of 0.1, each representing a 10\% change relative to the baseline.
        The y-axis represents the theta and gamma power (blue and orange, respectively).
    \end{minipage}
\end{figure}

\begin{table}[htbp]
    \centering
    \caption[Summary of Model validation: Network simulation Parameters and Results]{Overview of Model Validation Network Simulation Settings and Outcomes.
        Variations in the firing rates of cell populations, alongside theta and gamma oscillations within the local field potential, as well as the alterations in their intensity upon the decrease of dendritic inhibition and the concurrent enhancement of external stimuli to the pyramidal neurons.}\label{tab:validation_results}
    \begin{adjustbox}{width=\textwidth}
        \begin{tabular}{ccccccccc}
            \hline
            OLM-Pyr Wt & External Wt & \CellWithForcedBreak{Pyr (Hz)                                                     \\ + Std} & \CellWithForcedBreak{BWB (Hz) \\ + Std} & \CellWithForcedBreak{OLM (Hz) \\ + Std} & \CellWithForcedBreak{Theta \\ Freq (Hz)} & \CellWithForcedBreak{Theta power \\ (mV\textsuperscript{2} Hz\textsuperscript{-1})} & \CellWithForcedBreak{Gamma \\ Freq (Hz)} & \CellWithForcedBreak{Gamma power \\ (mV\textsuperscript{2} Hz\textsuperscript{-1})} \\
            \hline
            1.0X       & 1.0X        & 1.83±0.58                     & 9.21±2.20  & 1.08±0.38 & 6.2 & 1.67 & 32.8 & 0.93 \\
            0.9X       & 1.1X        & 2.00±0.64                     & 10.67±2.31 & 1.15±0.36 & 6.2 & 2.08 & 34.4 & 1.36 \\
            0.8X       & 1.2X        & 2.00±0.71                     & 11.54±2.72 & 1.22±0.39 & 6.2 & 1.67 & 34.4 & 1.74 \\
            0.7X       & 1.3X        & 2.14±0.74                     & 13.24±2.36 & 1.29±0.42 & 6.2 & 2.06 & 32.8 & 2.37 \\
            0.6X       & 1.4X        & 2.33±0.79                     & 15.04±2.35 & 1.44±0.45 & 6.2 & 2.13 & 32.8 & 2.76 \\
            0.5X       & 1.5X        & 2.41±0.80                     & 16.44±2.31 & 1.53±0.45 & 6.2 & 2.02 & 32.8 & 3.68 \\
            0.4X       & 1.6X        & 2.57±0.79                     & 17.88±2.12 & 1.76±0.45 & 6.2 & 2.03 & 31.2 & 4.44 \\
            0.3X       & 1.7X        & 2.74±0.86                     & 19.04±2.38 & 2.10±0.49 & 6.2 & 1.55 & 31.2 & 4.68 \\
            0.2X       & 1.8X        & 2.97±0.88                     & 20.61±2.14 & 2.44±0.48 & 6.2 & 0.92 & 31.2 & 4.87 \\
            0.1X       & 1.9X        & 3.59±0.99                     & 20.98±2.62 & 3.31±0.50 & 6.2 & 0.78 & 32.8 & 5.82 \\
            0.0X       & 2.0X        & 5.50±1.52                     & 21.46±7.93 & 4.90±0.52 & 4.7 & 0.04 & 39.1 & 1.87 \\
            \hline
        \end{tabular}
    \end{adjustbox}
\end{table}
\pagebreak
\section{Results of the Sodium-Potassium variants}

\todo[inline]{add some text for the firing rates plot of sodium potassium}


\begin{figure}[htbp]
    \centering
    \includegraphics[width=1.0\textwidth]{Cell_firing_rates_per_pop_per_variant.png}
    \caption[Sodium-Potassium variants: Firing rates per population]{Sodium-Potassium variants: Firing rates per population.}\label{fig:sodium_potassium_firing_rates}
    \begin{minipage}{0.9\textwidth}
        The above figure shows the firing rates of the Pyr cells, BC cells, and O-LM cells for each modified cell type.
        Each of the three cell types either had modified sodium or potassium conductance.
        The firing rates were calculated from the spike activity of the cells in each population for the duration of the simulation (5000 ms).
        The x-axis represents the percentage amount of changed sodium or potassium conductance, times the baseline.
        The y-axis represents the firing rate in Hz.
        The firing rates are per cell type: Pyr (blue), Basket (cyan) and OLM (red).
        The error bars represent the standard error of the mean (SEM) of the firing rates per population.
    \end{minipage}
\end{figure}

\begin{figure}[htbp]
    \centering
    \includegraphics[width=1.0\textwidth]{Theta_gamma_freqs_variants.png}
    \caption[Sodium-Potassium variants: Dominant frequencies]{Sodium-Potassium variants: Dominant frequencies.}\label{fig:sodium_potassium_frequencies}
    \begin{minipage}{0.9\textwidth}
        The above figure shows the dominant theta-gamma frequencies in the network activity for each modified cell type.
        Each of the three cell types either had modified sodium or potassium conductance (pyr = blue, OLM = red, basket = cyan).
        The dominant frequencies were calculated from the local field potential (LFP) for the duration of the simulation (5000 ms).
        The x-axis represents the percentage amount of changed sodium or potassium conductance, times the baseline.
        The y-axis represents the dominant frequency in Hz for both theta (3--12 Hz, blue) and gamma (30--80 Hz, orange) oscillatory bands.
    \end{minipage}
\end{figure}
\begin{figure}[htbp]
    \centering
    \includegraphics[width=1.0\textwidth]{Theta_gamma_power_variants.png}
    \caption[Sodium-Potassium variants: Theta-Gamma power]{Sodium-Potassium variants: Theta-Gamma power.}\label{fig:sodium_potassium_power}
    \begin{minipage}{0.9\textwidth}
        The above figure shows the power of the theta and gamma oscillations in the network for each modified cell type.
        Each of the three cell types either had modified sodium or potassium conductance (pyr = blue, OLM = red, basket = cyan).
        The power of the theta and gamma oscillations were calculated from the local field potential (LFP) for the duration of the simulation (5000 ms).
        The x-axis represents the percentage amount of changed sodium or potassium conductance, times the baseline.
        The y-axis represents the theta and gamma power (blue and orange, respectively).
    \end{minipage}
\end{figure}
\pagebreak

\section{Results of the External Noise variants}

\todo[inline]{don't forget to add some text to the previous results sections, not just captions under figures.}

\begin{figure}[H]
    \centering
    % Row 1
    \begin{subfigure}{0.48\textwidth}
        \includegraphics[width=\linewidth]{DPB_percentage_matrices/DPB_percentage_noise_0.65.png}
        \caption{} % Optional caption
    \end{subfigure}\hfill
    \begin{subfigure}{0.48\textwidth}
        \includegraphics[width=\linewidth]{DPB_percentage_matrices/DPB_percentage_noise_0.75.png}
        \caption{} % Optional caption
    \end{subfigure}

    \bigskip % Adds vertical space between the rows

    % Row 2
    \begin{subfigure}{0.48\textwidth}
        \includegraphics[width=\linewidth]{DPB_percentage_matrices/DPB_percentage_noise_0.85.png}
        \caption{} % Optional caption
    \end{subfigure}\hfill
    \begin{subfigure}{0.48\textwidth}
        \includegraphics[width=\linewidth]{DPB_percentage_matrices/DPB_percentage_noise_0.95.png}
        \caption{} % Optional caption
    \end{subfigure}

    \caption[DPB percentage matrices]{Percentage of trials where depolarization block events occurred for all tested noise conditions.
    The x-axis shows all the sodium conductance changes in pyramidal cells, whereas the y-axis shows the potassium conductance changes in pyramidal cells.
    Modifications to pyramidal cells were applied to all compartments.
    The color intensity scales from 0 to 100 \%, where high-intensity yellow equals a higher amount of DPB events in a condition.
    The images are labeled from low noise to higher noise (a through d), respectively.}\label{fig:dpb_percentage_matrices}
\end{figure}

\begin{figure}[H]
    \centering
    % Row 1
    \begin{subfigure}{0.48\textwidth}
        \includegraphics[width=\linewidth]{DPB_delay_matrices/DPB_delay_noise_0.65.png}
        \caption{} % Optional caption
    \end{subfigure}\hfill
    \begin{subfigure}{0.48\textwidth}
        \includegraphics[width=\linewidth]{DPB_delay_matrices/DPB_delay_noise_0.75.png}
        \caption{} % Optional caption
    \end{subfigure}

    \bigskip % Adds vertical space between the rows

    % Row 2
    \begin{subfigure}{0.48\textwidth}
        \includegraphics[width=\linewidth]{DPB_delay_matrices/DPB_delay_noise_0.85.png}
        \caption{} % Optional caption
    \end{subfigure}\hfill
    \begin{subfigure}{0.48\textwidth}
        \includegraphics[width=\linewidth]{DPB_delay_matrices/DPB_delay_noise_0.95.png}
        \caption{} % Optional caption
    \end{subfigure}

    \caption[DPB delay matrices]{Average delay + Standard deviation of DPB in trials where depolarization block events occurred for all tested noise conditions.
    The x-axis shows all the sodium conductance changes in pyramidal cells, whereas the y-axis shows the potassium conductance changes in pyramidal cells.
    Modifications to pyramidal cells were applied to all compartments.
    The color intensity shows the average delay, where high-intensity red equals a shorter delay in DPB events in a condition.
    The images are labeled from low noise to higher noise (a through d), respectively.}\label{fig:dpb_delay_matrices}
\end{figure}
\pagebreak
\section{Results of the External Noise: Burst analysis}
% First page with 3 sub-figures
\begin{figure}[!htb]
    \centering
    % First sub-figure
    \begin{subfigure}{\textwidth}
        \includegraphics[width=\linewidth]{network_activity_0.80_0.80_1.00.png}
        \caption{} % Optional caption or label
    \end{subfigure}
    \vspace{1em} % Space between the sub-figures

    % Second sub-figure
    \begin{subfigure}{\textwidth}
        \includegraphics[width=\linewidth]{network_activity_0.80_1.20_1.00.png}
        \caption{} % Optional caption or label
    \end{subfigure}
    \vspace{1em} % Space between the sub-figures

    % Third sub-figure
    \begin{subfigure}{\textwidth}
        \includegraphics[width=\linewidth]{network_activity_1.00_1.00_1.00.png}
        \caption{} % Optional caption or label
    \end{subfigure}

    \caption{Burst detection near the onset of the depolarization block. Shows the 2nd to last burst before the DPB (inter-ictal, left), 
    the last burst before the DPB (pre-ictal, middle), and the first burst after the DPB (ictal, right). 
    The bursts are aligned relative the the onset of the burst of the pyramidal population in the network (ms). 
    The curve of the spike burst shows the standard deviation of the spike activity as a filled gradient surrounding the curve. 
    The bursts of both pyramidal neurons (blue) and basket cells are shown (green).}\label{fig:burst_detection}
\end{figure}
\clearpage % Ensure the continuation starts on a new page

% Second page with 2 sub-figures continued
\begin{figure}[!htb] \ContinuedFloat% Continue the figure from the previous page
    \centering
    % Fourth sub-figure
    \begin{subfigure}{\textwidth}
        \includegraphics[width=\linewidth]{network_activity_1.20_0.80_1.00.png}
        \caption{} % Optional caption or label
    \end{subfigure}
    \vspace{1em} % Space between the sub-figures

    % Fifth sub-figure
    \begin{subfigure}{\textwidth}
        \includegraphics[width=\linewidth]{network_activity_1.20_1.20_1.00.png}
        \caption{} % Optional caption or label
    \end{subfigure}

    % No need for a new caption or label; it's a continuation
\end{figure}

\section{Results of the Recurrent Connections variants}
\begin{figure}[H]
    \centering
    % Row 1
    \begin{subfigure}{0.48\textwidth}
        \includegraphics[width=\linewidth]{DPB_RC_Percentages/DPB_percentages_RC_1.00.png}
        \caption{} % Optional caption
    \end{subfigure}\hfill
    \begin{subfigure}{0.48\textwidth}
        \includegraphics[width=\linewidth]{DPB_RC_Percentages/DPB_percentages_RC_1.05.png}
        \caption{} % Optional caption
    \end{subfigure}

    \bigskip % Adds vertical space between the rows

    % Row 2
    \begin{subfigure}{0.48\textwidth}
        \includegraphics[width=\linewidth]{DPB_RC_Percentages/DPB_percentages_RC_1.10.png}
        \caption{} % Optional caption
    \end{subfigure}\hfill
    \begin{subfigure}{0.48\textwidth}
        \includegraphics[width=\linewidth]{DPB_RC_Percentages/DPB_percentages_RC_1.15.png}
        \caption{} % Optional caption
    \end{subfigure}

    \caption[RC DPB percentage matrices (all)]{Percentage of trials where depolarization block events occurred for all tested noise conditions.
        The x-axis shows all the sodium conductance changes in pyramidal cells, whereas the y-axis shows the potassium conductance changes in pyramidal cells.
        Modifications to pyramidal cells were applied to all compartments.
        The color intensity shows the average delay, where high-intensity yellow equals higher percentage of DPB events in a condition.
        The images are labeled from low noise to higher noise (a through d), respectively.}\label{fig:rc_dpb_percentage_matrices}
\end{figure}