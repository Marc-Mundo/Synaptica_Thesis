\chapter{Introduction}


\section{Background}
Epilepsy is a neurological disorder characterized recurrent seizures.
Seizures have to be two or more unprovoked and more than 24 hours apart, a single unprovoked
seizure with a high recurrence risk (\(>\)60 \% over the next 10 years); or the patient needs to have been previously diagnosed with an epilepsy syndrome.
Patients with epilepsy usually also suffer cognitive challenges, language difficulties, and an increased risk of mental health issues,
such as anxiety and depression~\parencite{fisherILAEOfficialReport2014}.

At its core, epilepsy involves an imbalance between excitatory and inhibitory processes within the brain.
This imbalance can originate in specific brain regions and spread, affecting various interconnected areas
outside of the epileptogenic zone~\parencite{ludersEpileptogenicZoneGeneral2006}.
The minimum amount of brain tissue of an associated region that initiates a seizure is therefore not fixed and is the
reason epilepsy is often though of as a network disorder.

Temporal Lobe Epilepsy (TLE), the most prevalent form of focal epilepsy in adults (60\% of cases),
impacts over 30 million people globally by estimation of the World Health Organization.
TLE's epileptic events often begin in brain regions like the hippocampus and entorhinal cortex, known for their
capacity to independently produce epileptiform activities~\parencite{lyttonComputerSimulationEpilepsy2005}.
Despite improvements over the years, not all TLE patients respond to medication, are subject to drug-resistance (25--30\%) or
suffer intolerable adverse effects (30--40\%,~\textcite{hakamiEfficacyTolerabilityAntiseizure2021}). Thus, research into TLE's
underlying mechanisms is vital for developing effective new treatments.

The hippocampus, located in the temporal lobe, is crucial for memory formation
and retrieval, emotion regulation, and spatial navigation. In TLE, it often
serves as the seizure's focal point, especially its CA3 subfield. The CA3 region,
with its dense connections and low epilepsy activation threshold, is particularly
susceptible to hyper-excitability~\parencite{witterIntrinsicExtrinsicWiring2007}.
In TLE, seizures can be triggered by excessive input from higher cortical regions, such as the visual or auditory cortex
that project to the hippocampus via the entorhinal cortex.

How excessive input is handled however, is dependent on the network's intrinsic properties and its functional connectivity.
The CA3 subfield of the hippocampus plays a pivotal role in facilitating higher cognitive functions such as memory encoding and retrieval,
spatial navigation, and pattern completion and separation. The region exhibits robust capacity for synaptic plasticity via long-term potentiation
(LTP) and depression (LTD), mechanisms that underlie learning and memory and are modulated by the level of neural activity~\parencite{stokesComplementaryRolesHuman2015}.

Perturbations in the connectivity of the CA3 region are common in TLE and directly impacts the working memory (WM)
of the network~\parencite{arskiOscillatoryBasisWorking2021}. However, the exact mechanisms by which these perturbations lead to epileptiform
activity are not well understood. The hippocampus is particularly susceptible to connectivity variability which can be transiently induced by epileptic discharge.
This is likely due to the high processing demands for WM~\parencite{aldenkampEffectsEpileptiformEEG2004}.

The neural assemblies that constitute the circuits in the CA3 region are highly regulated and the transfer of
information propagates through specific oscillatory patterns.
Characteristic neural oscillations in the hippocampus, such as theta (3--12 Hz)
and gamma (30--80 Hz) rhythms, play significant roles in forming episodic memory and cognition~\parencite{nyhusFunctionalRoleGamma2010}.

Epilepsy is associated with alterations in Cross-Frequency Coupling (CFC), where
the phase of slower waves modulates the amplitude of faster waves, reflecting
disrupted network functionality. Abnormalities in Theta-Gamma Phase-Amplitude
Coupling (PAC) correlate with cognitive impairments in epilepsy. Therefore, tracking changes
in these neural oscillations can provide insights into the disease's progression.

Detecting the state of the brain is crucial for predicting and managing seizures, as patients of epilepsy
only experience the effects of their disease during seizures.
The gold standard of brain activity detection utilizes the electroencephalogram (EEG),
which can adequately measure electrical fluctuations in the range of theta-gamma oscillations.

The associated brain region of an epilepsy patient when investigated can be in one of three states that have been generally defined using
various epilepsy detection algorithms:

\begin{itemize}
    \item \textbf{Inter-ictal State}
          \begin{itemize}
              \item \textit{Definition}: The period between seizures, with no active seizure activity.
              \item \textit{Characterization}: Characterized by inter-ictal spikes or sharp waves in EEG\@.
                    These spikes indicate abnormal electrical discharges that are not actual seizures.
          \end{itemize}
    \item \textbf{Pre-ictal State}
          \begin{itemize}
              \item \textit{Definition}: The period immediately before the onset of a seizure.
              \item \textit{Characterization}: Marked by subtle and variable changes in EEG and other physiological signals that precede seizures,
                    crucial for seizure prediction efforts.
          \end{itemize}
    \item \textbf{Ictal State}
          \begin{itemize}
              \item \textit{Definition}: The period during which a seizure occurs.
              \item \textit{Characterization}: EEG shows sustained, rhythmic electrical activity distinct from normal or inter-ictal activity,
                    with corresponding behavioral symptoms based on seizure type.
          \end{itemize}
\end{itemize}

\noindent Classification methods can extract the aforementioned epileptic features from the EEG based on comparison of
different kernels (Linear, Sigmoid, Grid, etc) The classification accuracy of these methods vary but are fairly high (\(>\)98.9 \%) for methods such as
Support Vector Machine (SVM) or Wavelet Neural Network (WNN) classification~\parencite{yayikEpilepticStateDetection2015}.
These methods demonstrate significant potential in enhancing the accuracy of epilepsy state detection, offering promising avenues for more
effective monitoring and intervention strategies in epilepsy management.

Identification of the epileptic states provides a benchmark for potential epileptiform activity in a (simulated) network.
Epilepsy as a whole could be viewed particular brain functioning state, manifested in a multi-state network of coupled oscillatory systems.
Therefore, tracking observable phenomena in the CA3 region of the hippocampus, such as bursting or oscillatory coupling could provide insights
into the network's susceptibility to ictal transitions~\parencite{kalitzinEpilepsyManifestationMultistate2019a}.

\subsubsection{Sodium and Potassium Channels in Epilepsy}
In the past 20 years, research has shown that at least half of all epilepsy cases have a genetic basis.
Rapid discovery of disease-causing genes have identified genes encoding for ion channel proteins.
Remarkably, a quarter of all cases involving monogenic variants are related to ion channels~\parencite{strianoGeneticTestingPrecision2020,oyrerIonChannelsGenetic2018}.

Sodium and potassium channels in particular, are essential for maintaining the resting
membrane potential and action potential generation in neurons.
Dysfunctional variants of sodium or potassium channels can lead to hyperexcitability in neurons, depending on whether the relevant mutation causes loss or gain of function.
This in turn can have destabilizing effects on neural circuits in regions such as the CA3 subfield of the hippocampus in TLE\@.

A lot of epileptic mutations have been found in \textit{voltage-gated} type ion channel genes.
For example, for sodium these are most often related to the brain-expressed SCN family (\textit{SCN1A, SCN2A, SCN3A and SCN8A}, \textcite{brunklausSodiumChannelEpilepsies2020}).
Highly conserved across species, these genes encode for the pore-forming alpha subunits of the sodium channel and are responsible for the fast depolarizing current in neurons.
Previous research has shown that mutations in these genes have become increasingly frequent in Mendelian epilepsy syndromes in recent years~\parencite{brunklausSodiumChannelEpilepsies2020}.
These SCN genes are involved in generalized epilepsy with febrile seizures plus (GEFS+) syndrome.
SCN1A in particular encodes predominantly in inhibitory GABAergic neurons, usually enriched in axonal segments and facilitates initiation of action potentials~\parencite{yuReducedSodiumCurrent2006}.
Classically, these voltage-gated channels only allow for passage of ions transiently, and are closed at rest.
However, a fraction of sodium ions passes through constantly (1\%-2\%).
So while these channels form the foundation for excitable cell function, a small fraction of channels shows persistent current, \(\text{I}_{\text{NaP}}\).
This current is normally contributes to proper physiological function of neurons, but can become pathogenic when dysregulated.
In recent years \(\text{I}_{\text{NaP}}\) has been identified as an important factor in sodium channelopathies, which could provide targets for novel anti-seizure medication strategies~\parencite{wengertRolePersistentSodium2021}.

Similarly, there are also common mutations in potassium channels that are associated with epilepsy.
The KCNA family genes that encode the respective subunits in potassium channels (\textit{KCNA1 and KCNA2}, \textcite{gaoPotassiumChannelsEpilepsy2022})
are commonly involved in episodic ataxia type 1 (EA1, \textcite{gravesIonChannelsEpilepsy2006}).
The dysfunction of these channels can lead to hyperexcitability in neurons, where modified dynamics of voltage gated channels lead to changes in the action potential's shape and duration.
In healthy neurons regulation of current happens via inwardly rectifying potassium channels that maintain the resting membrane potential~\parencite{isomotoInwardlyRectifyingPotassium1997}.
In channelopathies regarding voltage-gated potassium channels, altered potassium gradients or accelerated hyperpolarization might occur which can no longer be compensated for by the rectifying
potassium channels~\parencite{nikitinPotassiumChannelsProminent2021}.

A lot of genetic variety exists for both of these major channels and can both lead to a destabilized network, hallmark of epileptic activity.
However, it remains difficult to quantitatively translate the genetic profile to network dynamics.

\subsubsection{Computational Modeling in Neuroscience with NEURON}
The NEURON simulator is a powerful tool for modeling the intricate dynamics
of neurons and their networks. It supports detailed simulations of membrane
dynamics, synaptic interactions, and the Hodgkin-Huxley model for action
potentials. The Hodgkin-Huxley model, a cornerstone of computational neuroscience,
describes the ionic currents underlying action potentials in neurons based
on non-linear differential equations~\parencite{hodgkinMeasurementCurrentvoltageRelations1952}.
This simulation framework allows for quite realistic representation of life-like neurons based
on experimentally-defined cellular characteristics. Many of which are freely available on ModelDB,
but vary in terms of complexity such as morphology, synapses or 3D-micro-environment.

NEURON's Python interface facilitates scripting and integration
with other Python-based tools, enhancing its utility in neuroscience research.
This simulator is crucial for studying neural phenomena, the effects of
anti-epileptic drugs, and diseases caused by dysfunctional ion channels for sodium and potassium,
providing invaluable insights into the functioning of the nervous system
and the development of therapeutic strategies~\parencite{miglioreParallelNetworkSimulations2006}.
\pagebreak
\section{Aim of the research}
This study aims to further investigate the role of the CA3 region of the hippocampus in epilepsy,
by adapting the computational model of the CA3 region of the hippocampus by~\textcite{neymotinKetamineDisruptsTheta2011} in the NEURON simulator.
The model consists of 800 pyramidal, 200 O-LM interneurons and 200 basket cells.
The baseline model contains enough biophysical detail to replicate homeostatic neural activity, consisting of theta-modulated gamma oscillations.

This research builds upon experiments by~\textcite{sanjayImpairedDendriticInhibition2015},
which focussed on reducing dendritic inhibition, increasing external stimulation and modifying synaptic connectivity as potential causes for TLE\@.

Instead, this study will focus on the role of sodium and potassium channels via ion-conductance modifications throughout the network,
the sensitivity for external noise in such conditions and the effects increasing recurrent connectivity of inhibitory basket cells.
\pagebreak
\section{Research question}
This research is focused on on the following research question:

\subsection*{Main Research Question}

\begin{itemize}
    \item What is the impact of different genetic profiles on epileptic activity in simulated hippocampal brain circuits?
\end{itemize}

\subsection*{Main Hypothesis}

\begin{itemize}
    \item It is hypothesized that by emulating specific genetic profiles within simulated CA3 hippocampal neural networks,
          we can replicate their impact on the networks' behavior. Furthermore, it is expected that inducing cellular dynamics that
          reduce seizure-like (ictal) activity will rescue normal network function.
\end{itemize}

\subsection*{Investigative Sub-Questions}

\begin{enumerate}
    \item What specific variables characterize (inter)-ictal activity in different neural cell types within temporal lobe epilepsy (TLE) networks?
    \item How do these variables vary across different neural networks, and how can they be reliably measured and quantified in a NEURON model?
    \item How can genetic variations, such as single nucleotide polymorphisms (SNPs) or known epilepsy-inducing mutations, be successfully integrated into hippocampal neural network models to simulate the impact on ictal activity?
    \item What alterations in network dynamics and connectivity patterns are observed when introducing epilepsy-associated genetic variations into neural cell types?
    \item Can the effects of specific genetic variants be correlated with the severity and frequency of seizures in the modeled networks?
    \item Can we identify specific targets for intervention that can restore normal network dynamics in the presence of epilepsy-associated genetic variations?
    \item Are we able to reliable predict the effect of a specific drug on the network dynamics and emulate them in the model?
\end{enumerate}