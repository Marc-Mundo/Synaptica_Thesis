\chapter{Introduction}

\todo[inline, color=red!40]{add some info to the empty subsection, maybe reorder some sections.}
\todo[inline, color=red!40]{see the introduction-outline.md file to see what to do for this chapter.}
\todo[inline]{ADD REFERENCES TO THE TEXT}

\section{Background}
In this chapter, the research topic is introduced. 
The chapter starts with a brief overview of epilepsy, followed by a discussion on the role of the hippocampus in temporal lobe epilepsy. 
The chapter then discusses the role of the CA3 region of the hippocampus in epilepsy. 
The chapter then discusses the role of computational modeling in neuroscience. 
Finally, the chapter discusses the literature gap that this research aims to address.

\subsection{Introduction to Epilepsy and Its Complexity}
Epilepsy is a neurological disorder characterized recurrent seizures. 
Seizures have to be two or more unprovoked and more than 24 hours apart, a single unprovoked
seizure with a high recurrence risk (>60 \% over the next 10 years); or the patient needs to have been previously diagnosed with an epilepsy syndrome.
Patients with epilepsy usually also suffer cognitive challenges, language difficulties, and an increased risk of mental health issues, 
such as anxiety and depression~\parencite{fisherILAEOfficialReport2014}. At its core, epilepsy 
involves an imbalance between excitatory and inhibitory processes within the brain. 
This imbalance can originate in specific brain regions and spread, affecting various interconnected areas 
outside of the epileptogenic zone~\parencite{ludersEpileptogenicZoneGeneral2006}.
The minimum amount of brain tissue of an associated region that initiates a seizure is therefore not fixed and is the 
reason epilepsy is often though of as a network disorder.

\subsection{Temporal Lobe Epilepsy: A Closer Look}
Temporal Lobe Epilepsy (TLE), the most prevalent form of focal epilepsy in adults, 
impacts over 50 million people globally. TLE's epileptic events often begin in 
brain regions like the hippocampus and entorhinal cortex, known for their 
capacity to independently produce epileptiform activities~\parencite{lyttonComputerSimulationEpilepsy2005}.
Despite many TLE patients not responding to medication, research into TLE's 
underlying mechanisms is vital for developing new treatments.

\subsection{The Hippocampus and Its Role in TLE}
The hippocampus, located in the temporal lobe, is crucial for memory formation 
and retrieval, emotion regulation, and spatial navigation. In TLE, it often 
serves as the seizure's focal point, especially its CA3 subfield. The CA3 region, 
with its dense connections and low epilepsy activation threshold, is particularly 
susceptible to hyper-excitability~\parencite{witterIntrinsicExtrinsicWiring2007}. 
Seizures can be triggered by excessive stimulation from higher cortical regions, such as the visual or auditory cortex
that project to the hippocampus via the entorhinal cortex in TLE\@.

\subsection{Functional Brain Networks and Oscillations}
How excessive stimulation is handled however, is dependent on the network's intrinsic properties and its functional connectivity.
The CA3 subfield of the hippocampus plays a pivotal role in facilitating higher cognitive function. 
Perturbations in the connectivity of the CA3 region are common in TLE and directly impacts the working memory (WM)
of the network~\parencite{arskiOscillatoryBasisWorking2021}. However, the exact mechanisms by which these perturbations lead to epileptiform 
activity are not well understood. The hippocampus is particularly susceptible to connectivity variability which can be transiently induced by epileptic discharge. 
This is likely due to the high processing demands for WM~\parencite{aldenkampEffectsEpileptiformEEG2004}.

The neural assemblies that constitute the circuits in the CA3 region are highly regulated and the transfer of 
information propagates through specific oscillatory patterns. 
Characteristic neural oscillations in the hippocampus, such as theta (3--12 Hz) 
and gamma (30--80 Hz) rhythms, play significant roles in forming episodic memory and cognition~\parencite{nyhusFunctionalRoleGamma2010}. 
Epilepsy is associated with alterations in Cross-Frequency Coupling (CFC), where 
the phase of slower waves modulates the amplitude of faster waves, reflecting 
disrupted network functionality. Abnormalities in Theta-Gamma Phase-Amplitude 
Coupling (PAC) correlate with cognitive impairments in epilepsy. Therefore, tracking changes
in these neural oscillations can provide insights into the disease's progression.

\subsection{Epileptic States}
Detecting the state of the brain is crucial for predicting and managing seizures, as patients of epilepsy
only experience the effects of their disease during seizures.
The gold standard of brain activity detection utilizes the electroencephalogram (EEG), 
which measures electrical fluctuations ranging from < 1 Hz to several kHz.

The associated brain region of an epilepsy patient when investigated can be in one of three states that have been generally defined using 
various epilepsy detection algorithms. These classification methods extract epileptic features from the EEG based on comparison of 
different kernels (Linear, Sigmoid, Grid, etc) The classification accuracy of these methods vary but are fairly high (>98.9 \%) for methods such as
Support Vector Machine (SVM) or Wavelet Neural Network (WNN) classification~\parencite{yayikEpilepticStateDetection2015}.
\pagebreak

These general epilepsy states for a patient are defined as follows:
\begin{itemize}
    \item \textbf{Inter-ictal State}
    \begin{itemize}
        \item \textit{Definition}: The period between seizures, with no active seizure activity.
        \item \textit{Characterization}: Characterized by inter-ictal spikes or sharp waves in EEG\@.
        These spikes indicate abnormal electrical discharges that are not actual seizures.
    \end{itemize}
    \item \textbf{Ictal State}
    \begin{itemize}
        \item \textit{Definition}: The period during which a seizure occurs.
        \item \textit{Characterization}: EEG shows sustained, rhythmic electrical activity distinct from normal or inter-ictal activity, 
        with corresponding behavioral symptoms based on seizure type.
    \end{itemize}
    \item \textbf{Pre-ictal State}
    \begin{itemize}
        \item \textit{Definition}: The period immediately before the onset of a seizure.
        \item \textit{Characterization}: Marked by subtle and variable changes in EEG and other physiological signals that precede seizures, 
        crucial for seizure prediction efforts.
    \end{itemize}
\end{itemize}

\noindent
Identification of the aforementioned states provides a benchmark for potential epileptiform activity in a (simulated) network.
Epilepsy as a whole could be viewed particular brain functioning state, manifested in a multi-state network of coupled oscillatory systems.
Thus tracking observable phenomena in the CA3 region of the hippocampus, such as bursting or oscillatory coupling could provide insights 
into the network's susceptibility to ictal transitions~\parencite{kalitzinEpilepsyManifestationMultistate2019a}.

\subsection{Sodium and Potassium Channels in Epilepsy}
In the past 20 years, research has shown that at least half of all epilepsy cases have a genetic basis.
Rapid discovery of disease-causing genes have identified genes encoding for ion channel proteins.
Remarkably, a quarter of all cases involving monogenic variants are related to ion channels~\parencite{strianoGeneticTestingPrecision2020,oyrerIonChannelsGenetic2018}.
Sodium and potassium channels in particular, are essential for maintaining the resting
membrane potential and action potential generation in neurons.
Dysfunctional variants of sodium or potassium channels can lead to hyperexcitability in neurons, depending on whether the relevant mutation causes loss or gain of function.
This in turn can have destabilizing effects on neural circuits in regions such as the CA3 subfield of the hippocampus in TLE\@.

A lot of epileptic mutations have been found in \textit{voltage-gated} type ion channel genes.
For sodium these are most often related to the brain-expressed SCN family (\textit{SCN1A, SCN2A, SCN3A and SCN8A}, \textcite{brunklausSodiumChannelEpilepsies2020}), 
or KCNA family for potassium  (\textit{KCNA1 and KCNA2}, \textcite{gaoPotassiumChannelsEpilepsy2022}).
The SCN gene family is involved in  generalized epilepsy with febrile seizures plus (GEFS+) syndrome, while KCNA in episodic ataxia type 1 (EA1, \textcite{gravesIonChannelsEpilepsy2006}).
\todo[inline]{add more info on the role of these channels in epilepsy.}

\subsection{Computational Modeling in Neuroscience with NEURON}
The NEURON simulator is a powerful tool for modeling the intricate dynamics 
of neurons and their networks. It supports detailed simulations of membrane 
dynamics, synaptic interactions, and the Hodgkin-Huxley model for action 
potentials. The Hodgkin-Huxley model, a cornerstone of computational neuroscience,
describes the ionic currents underlying action potentials in neurons based
on non-linear differential equations~\parencite{hodgkinMeasurementCurrentvoltageRelations1952}.

NEURON's Python interface facilitates scripting and integration 
with other Python-based tools, enhancing its utility in neuroscience research.
This simulator is crucial for studying neural phenomena, the effects of 
anti-epileptic drugs, and diseases caused by dysfunctional ion channels for sodium and potassium, 
providing invaluable insights into the functioning of the nervous system 
and the development of therapeutic strategies~\parencite{miglioreParallelNetworkSimulations2006}.

\subsection{Aim of the research}
This study aims to further investigate the role of the CA3 region of the hippocampus in epilepsy, 
by adapting the computational model of the CA3 region of the hippocampus by~\textcite{neymotinKetamineDisruptsTheta2011} in the NEURON simulator.
The model consists of 800 pyramidal, 200 O-LM interneurons and 200 basket cells. 
The baseline model contains enough biophysical detail to replicate homeostatic neural activity, consisting of theta-modulated gamma oscillations.

This research builds upon experiments by~\textcite{sanjayImpairedDendriticInhibition2015}, 
which focussed on reducing dendritic inhibition, increasing external stimulation and modifying synaptic connectivity as potential causes for TLE\@.

Instead, this study will focus on the role of sodium and potassium channels via ion-conductance modifications throughout the network, 
the sensitivity for external noise in such conditions and the effects varied recurrent connectivity.

\section{Research question}
Insert research questions here.