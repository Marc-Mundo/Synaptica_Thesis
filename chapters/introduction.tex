\chapter{Introduction}

\todo[inline, color=red!40]{add some info to the empty subsection, maybe reorder some sections.}
\todo[inline, color=red!40]{see the introduction-outline.md file to see what to do for this chapter.}
\todo[inline]{ADD REFERENCES TO THE TEXT}

\section{Background}
In this chapter, the research topic is introduced. 
The chapter starts with a brief overview of epilepsy, followed by a discussion on the role of the hippocampus in temporal lobe epilepsy. 
The chapter then discusses the role of the CA3 region of the hippocampus in epilepsy. 
The chapter then discusses the role of computational modeling in neuroscience. 
Finally, the chapter discusses the literature gap that this research aims to address.

\subsection{Introduction to Epilepsy and Its Complexity}
Epilepsy is a neurological disorder characterized recurrent seizures. 
Seizures have to be two or more unprovoked and more than 24 hours apart, a single unprovoked
seizure with a high recurrence risk (>60 \% over the next 10 years); or the patient needs to have been previously diagnosed with an epilepsy syndrome.
Patients with epilepsy usually also suffer cognitive challenges, language difficulties, and an increased risk of mental health issues, 
such as anxiety and depression~\parencite{fisherILAEOfficialReport2014}. At its core, epilepsy 
involves an imbalance between excitatory and inhibitory processes within the brain. 
This imbalance can originate in specific brain regions and spread, affecting various interconnected areas 
outside of the epileptogenic zone~\parencite{ludersEpileptogenicZoneGeneral2006}.
The minimum amount of brain tissue of an associated region that initiates a seizure is therefore not fixed and is the 
reason epilepsy is often though of as a network disorder.

\subsection{Temporal Lobe Epilepsy: A Closer Look}
Temporal Lobe Epilepsy (TLE), the most prevalent form of focal epilepsy in adults, 
impacts over 50 million people globally. TLE's epileptic events often begin in 
brain regions like the hippocampus and entorhinal cortex, known for their 
capacity to independently produce epileptiform activities~\parencite{lyttonComputerSimulationEpilepsy2005}.
Despite many TLE patients not responding to medication, research into TLE's 
underlying mechanisms is vital for developing new treatments.

\subsection{The Hippocampus and Its Role in TLE}
The hippocampus, located in the temporal lobe, is crucial for memory formation 
and retrieval, emotion regulation, and spatial navigation. In TLE, it often 
serves as the seizure's focal point, especially its CA3 subfield. The CA3 region, 
with its dense connections and low epilepsy activation threshold, is particularly 
susceptible to hyper-excitability~\parencite{witterIntrinsicExtrinsicWiring2007}. Seizures can be 
triggered by excessive stimulation from the entorhinal cortex, highlighting the 
complex interplay between different brain regions in TLE\@.

\subsection{Functional Brain Networks in Epilepsy}


\subsection{Neural Oscillations and Epilepsy}
Characteristic neural oscillations in the hippocampus, such as theta (3--12 Hz) 
and gamma (30--80 Hz) rhythms, play significant roles in memory and cognition. 
Epilepsy is associated with alterations in Cross-Frequency Coupling (CFC), where 
the phase of slower waves modulates the amplitude of faster waves, reflecting 
disrupted network functionality. Abnormalities in Theta-Gamma Phase-Amplitude 
Coupling (PAC) correlate with cognitive impairments in epilepsy, underscoring 
the importance of understanding these oscillations in neurological disorders.

\subsection{Interictal Network Dynamics}
The interictal state, marking intervals between seizures, reveals sporadic 
spiking in local field potentials (LFPs) indicative of abnormal electrical 
activity. These intermittent bursts highlight the underlying dysfunction in 
neural processing characteristic of epileptic networks, even outside of 
seizure episodes.

\subsection{Sodium and Potassium Channels in Epilepsy}
In the past 20 years, research has shown that at least half of all epilepsy cases have a genetic basis.
A quarter of all cases involve monogenic variants related to ion channels~\parencite{strianoGeneticTestingPrecision2020,oyrerIonChannelsGenetic2018}.
Sodium and potassium channels in particular, are essential for maintaining the resting
membrane potential and action potential generation in neurons.
Dysfunctional variants of sodium or potassium channels can lead to hyperexcitability in neurons and depending on their loss or gain of function.
This in turn can have destabilizing effects on neural circuits.
A lot of epileptic mutations have been found in voltage-gated ion channel genes.
For sodium these are most often related to the brain-expressed SCN family (\textit{SCN1A, SCN2A, SCN3A and SCN8A}, \textcite{brunklausSodiumChannelEpilepsies2020}), 
while potassium it usually KCNA (\textit{KCNA1 and KCNA2}, \textcite{gaoPotassiumChannelsEpilepsy2022}).
\todo[inline]{add some more info on these specific channels, check the references in the above subsection!}

\subsection{Computational Modeling in Neuroscience with NEURON}
The NEURON simulator is a powerful tool for modeling the intricate dynamics 
of neurons and their networks. It supports detailed simulations of membrane 
dynamics, synaptic interactions, and the Hodgkin-Huxley model for action 
potentials. NEURON's Python interface facilitates scripting and integration 
with other Python-based tools, enhancing its utility in neuroscience research.
This simulator is crucial for studying neural phenomena, the effects of 
anti-epileptic drugs, and diseases caused by dysfunctional ion channels, 
providing invaluable insights into the functioning of the nervous system 
and the development of therapeutic strategies.

\subsection{Aim of the research}
This study aims to further investigate the role of the CA3 region of the hippocampus in epilepsy, 
by adapting the computational model of the CA3 region of the hippocampus by~\textcite{neymotinKetamineDisruptsTheta2011} in the NEURON simulator.
The model consists of 800 pyramidal, 200 O-LM interneurons and 200 basket cells. 
The baseline model contains enough biophysical detail to replicate homeostatic neural activity, consisting of theta-modulated gamma oscillations.

This research builds upon experiments by~\textcite{sanjayImpairedDendriticInhibition2015}, 
which focussed on reducing dendritic inhibition, increasing external stimulation and modifying synaptic connectivity as potential causes for TLE\@.

Instead, this study will focus on the role of sodium and potassium channels via ion-conductance modifications throughout the network, 
the sensitivity for external noise in such conditions and the effects varied recurrent connectivity.

\subsection{Literature Gap}

\section{Research question}

\section{Outline of the thesis}