\chapter{Introduction}

\todo[inline, color=red!40]{see the introduction-outline.md file to see what to do for this chapter.}
\todo[inline]{ADD REFERENCES TO THE TEXT}

\section{Background}
In this chapter, the research topic is introduced. 
The chapter starts with a brief overview of epilepsy, followed by a discussion on the role of the hippocampus in temporal lobe epilepsy. 
The chapter then discusses the role of the CA3 region of the hippocampus in epilepsy. 
The chapter then discusses the role of computational modeling in neuroscience. 
Finally, the chapter discusses the literature gap that this research aims to address.

\subsection{Temporal Lobe Epilepsy and Neural Dynamics in the Hippocampus}
Epilepsy is a neurological disorder characterized by recurrent seizures, cognitive deficits, language disturbances and higher rates of psychological conditions including anxiety and depression.
Usually epilepsy is caused by the imbalance of excitatory and inhibitory neural activity in the brain.

While the point of origin of such an imbalance is often located within a specific region of the brain,
the effects of such imbalance can propagate through the brain to many other regions and cause seizures.
This makes epilepsy a complex disorder that is often regarded as a brain network disease.

Temporal lobe epilepsy (TLE) is the most common form of epilepsy in adults with over 50 million patients world-wide, and it is characterized by seizures that originate in the temporal lobe of the brain.

\subsection{Hippocampal CA3 Region's Role in Epilepsy}
The hippocampus is a region of the brain that is located in the temporal lobe and it is known to play a crucial role in the formation and retrieval of memories, while also involved in the regulation of emotions and spatial navigation.
In TLE the hippocampus is the typical focal point of the seizures, with the the CA3 subfield being the most common point of origin.

The CA3 region is a densely connected region of the hippocampus, with a high number of recurrent connections between its neurons.
Consequently, this region also has the lowest activation threshold for epilepsy of the hippocampus and is thus a hyperexcitable region.

Often, in TLE, epileptic episodes are initiated by high external stimulation from the entorhinal cortex.
The entorhinal cortex funnels stimuli from centers of the brain such as the visual cortex and the auditory cortex.
However, the exact mechanism of initiation is poorly understood.

\subsection{Theta-gamma oscillations in the hippocampus}
The hippocampus is known to exhibit characteristic neural oscillations, which are of a specific rhythm and frequency.
These oscillations are thought to be important for the formation of memories and the retrieval of memories.

The theta (3--12 Hz) and gamma (30--80 Hz) oscillations are two such oscillations that are known to be important in homeostatic processes.
Theta rhythms are linked to navigation and spatial awareness, while gamma rhythms are associated with processing and encoding of memory. 

In epilepsy, there's often an alteration in Cross-Frequency Coupling (CFC), a phenomenon where the phase of a slower frequency wave (like theta) modulates the amplitude of a faster frequency wave (like gamma), indicating disrupted brain network functionality typical in epilepsy. 
Theta-Gamma Phase-Amplitude Coupling (PAC), a specific type of CFC where the phase of theta waves influences the amplitude of gamma waves, is important for cognitive processes, with abnormalities in PAC correlating with impairments seen in epilepsy. 

Each brain frequency rhythm is associated with specific functionalities, such as theta rhythms with navigation, and gamma rhythms with memory processing. 
Disruptions in these rhythms can affect related cognitive functions, as seen in various neurological conditions including epilepsy.

\subsection{(Inter)-ictal network states in the hippocampus}
Intervals between hyper-synchronous events refer to the periods of time separating episodes of excessive synchronous neural activity, characteristic of seizure activity in epilepsy. 
Abnormal brain activity patterns typically represent less severe deviations from normal brain function, which may not always culminate in full-blown seizures but do indicate an underlying dysfunction in neural processing. 
Sporadic spiking in local field potential (LFP) or network activity, where the LFP reflects the flow of electric current within the brain tissue, suggests intermittent bursts of abnormal electrical activity, a hallmark of epileptic networks.

\subsection{Computational Modeling in Neuroscience}
The NEURON simulator excels in the detailed modeling of individual neurons and their networks, incorporating electro-physiological properties such as membrane dynamics, synaptic interactions, and signal integration within neural circuits. 
Its capability extends to the Hodgkin-Huxley ion channel model, implementable in a ``Ball \& Stick'' model for simulating action potentials, which is fundamental to understanding neuronal excitability.
NEURON can also simulate complex, nonlinear behaviors of neurons, a necessity for modeling real neuronal behavior and investigating a wide range of neural phenomena. 
Furthermore, NEURON's Python interface enhances its accessibility, allowing for scripting, automation of simulations, and integration with other Python-based tools for sophisticated model development, data analysis, and visualization.

NEURON is adept at simulating neural dynamics, including action potential generation and propagation, resting membrane potentials, and responses to stimuli. 
It enables the modeling of synaptic interactions, representing various types of synapses and their plasticity, to understand how neurons form functional networks. 
In addition to single-neuron models, NEURON can simulate large networks to study information processing and complex behaviors. 

It is also valuable for simulating the effects of anti-epileptic drugs (AEDs) on neuronal excitability and synaptic transmission, aiding in therapeutic development. 
Furthermore, NEURON can model channelopathies, diseases caused by dysfunctional ion channels, to explore their impact on neuronal behavior. 

In essence, the NEURON simulator is an indispensable tool for a broad spectrum of neuroscience investigations, offering insights into the normal and pathological workings of the nervous system, therapeutic strategies, and the molecular basis of neurological diseases.

\subsection{Literature Gap}

\section{Research question}

\section{Aim of the research}

\section{Outline of the thesis}